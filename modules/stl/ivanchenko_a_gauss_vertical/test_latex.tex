\documentclass{article}
\usepackage{graphicx} % Required for inserting images
\usepackage[warn]{mathtext}
\usepackage[T2A]{fontenc}
\usepackage[utf8]{luainputenc}
\usepackage[english, russian]{babel}
\usepackage[pdfpagemode=UseNone,colorlinks,allcolors=black]{hyperref}
\usepackage{tempora}
\usepackage[12pt]{extsizes}
\usepackage{listings}
\usepackage{color}
\usepackage{geometry}
\usepackage{enumitem}
\usepackage{multirow}
\usepackage{graphicx}
\usepackage{indentfirst}
\usepackage{amsmath}
\usepackage{soul}
\usepackage{pgfplots}
\pgfplotsset{compat=1.9}

\usepackage{listings}
\usepackage{color}

\definecolor{dkgreen}{rgb}{0,0.6,0}
\definecolor{gray}{rgb}{0.5,0.5,0.5}
\definecolor{mauve}{rgb}{0.58,0,0.82}

\lstset{frame=tb,
  language=C++,
  aboveskip=3mm,
  belowskip=3mm,
  showstringspaces=false,
  columns=flexible,
  basicstyle={\small\ttfamily},
  numbers=none,
  numberstyle=\tiny\color{gray},
  keywordstyle=\color{blue},
  commentstyle=\color{dkgreen},
  stringstyle=\color{mauve},
  breaklines=true,
  breakatwhitespace=true,
  tabsize=3
}
\begin{document}


\begin{titlepage}

\begin{center}
Министерство науки и высшего образования Российской Федерации
\end{center}

\begin{center}
Федеральное государственное автономное образовательное учреждение высшего образования \\
Национальный исследовательский Нижегородский государственный университет им. Н.И. Лобачевского
\end{center}

\begin{center}
Институт информационных технологий, математики и механики
\end{center}

\vspace{4em}

\begin{center}
\textbf{\LargeОтчет по лабораторной работе} \\
\end{center}
\begin{center}
\textbf{\Large<<Линейная фильтрация изображений - фильтр Гаусса} \\
\end{center}

\vspace{4em}

\newbox{\lbox}
\savebox{\lbox}{\hbox{text}}
\newlength{\maxl}
\setlength{\maxl}{\wd\lbox}
\hfill\parbox{7cm}{
\hspace*{5cm}\hspace*{-5cm}\textbf{Выполнил:} \\ студент группы 3821Б1ПМоп3\\Иванченко А. М.\\
\\
\hspace*{5cm}\hspace*{-5cm}\textbf{Проверил:}\\ младший научный сотрудник\\Нестеров А. Ю.\\
}
\vspace{\fill}

\begin{center} Нижний Новгород \\ 2024 \end{center}

\end{titlepage}
\setcounter{page}{2}

% Содержание
\tableofcontents
\newpage

% Введение
\section*{Введение}
\addcontentsline{toc}{section}{Введение}
\par
Значительную долю полученной информации об окружающем мире
люди получают при помощи зрительного восприятия, поэтому область компьютерной графики и компьютерного зрения очень важна и находит применение в множестве задач.
\par
Компьютерная графика - область науки и технологий, изучающая
создание, способы хранения, обработки и распознавания изображений, 3D
научной визуализации и визуализации 3D-сцен виртуальной реальности на
компьютере. Одна из важнейших областей компьютерной графики - обработка 2D изображений. Важной задачей является задача размытия изображения.
\par
Размытие по Гауссу в цифровой обработке изображений — способ размытия изображения с помощью функции Гаусса, названной в честь немецкого математика Карла Фридриха Гаусса.
\par
Этот эффект широко используется в графических редакторах для уменьшения шума изображения и снижения детализации. Визуальный эффект этого способа размытия напоминает эффект просмотра изображения через полупрозрачный экран, и отчётливо отличается от эффекта боке, создаваемого расфокусированным объективом или тенью объекта при обычном освещении. Размытие по Гауссу также используется в качестве этапа предварительной обработки в алгоритмах компьютерного зрения для улучшения структуры изображения в различных масштабах. 
\par
Фильтр Гаусса относится к так называемым матричным фильтрам. Применение таких фильтров к изображению - вычислительно сложная операция, поэтому увеличение производительности такого алгоритма с помощью использования параллельных вычислений является очень важной и актуальной задачей.
\newpage

% Постановка задачи
\section*{Постановка задачи}
\addcontentsline{toc}{section}{Постановка задачи}
%  Постановка задачи размытия
\subsection*{Размытие по Гауссу}
\addcontentsline{toc}{subsection}{Размытие по Гауссу}

\par
Содержательная задача фильтра Гаусса состоит в получении из исходного изображения изображение с эффектом размытия. Такой эффект можно достичь разными методами, и особенностью фильтра Гаусса является использование функции Гаусса для создания эффекта размытия.
\par
Функция Гаусса используется для создания матрицы фильтра (ядра). Поэтому для создания эффекта размытия требуется выполнить две задачи: создание матрицы фильтра (ядра) с заданным размером и применение фильтра к изображению.
% Техническое задание
\subsection*{Техническое задание}
\addcontentsline{toc}{subsection}{Техническое задание}
\par В рамках данной работы необходимо выполнить следующие задачи:
\begin{enumerate}
    \item Реализовать применение фильтра Гаусса с ядром 3x3 к изображению произвольного размера с использованием языка программирования C++
    \item Добавить параллельную реализацию метода на языке программирования C++ с использованием разных технологий написания параллельных программ для с общей памятью: OpenMP, TBB, std::threads.
    \item Сравнить результаты и время работы разных реализаций алгоритма на различных наборах данных. Оценить полученное ускорение для различных параллельных версий.
\end{enumerate}
\newpage

\section*{Реализция}
\addcontentsline{toc}{section}{Реализация}
% Описание алгоритма
\subsection*{Фильтр Гаусса}
\addcontentsline{toc}{subsection}{Фильтр Гаусса}
\par
В практике цифровой обработки изображений широко используется
масочная фильтрация. Ее линейная разновидность является одним из
вариантов двумерной фильтрации с конечной импульсной
характеристикой (КИХ) фильтра. В качестве маски используется
множество весовых коэффициентов, заданных во всех точках окрестности
$S$, обычно симметрично окружающих текущую точку кадра. К таким фильтрам относится и фильтр Гаусса.
Распространенным видом окрестности, часто применяемым на практике,
является квадрат 3 × 3 с текущим элементом в центре. Именно такую окрестность мы будем использовать в нашей реализации.
\par
Пространственная фильтрация определяется как операция двумерной свертки импульсной характеристики фильтра $h(s, t)$ с изображением $f(x, y)$, где $s$ - координата характеристики в горизонтальном направлении вдоль оси $x$, $t$ - координата характиристики в вертикальном направлении вдоль оси $y$:
$$g(x,y) = \sum_{t = -m/2}^{m/2} \sum_{s = -n/2}^{n/2} f(s,t) h(x - s, y - t) = 
\\
\sum_{t = -m/2}^{m/2} \sum_{s = -n/2}^{n/2} f(x - s, y - t) h(s, t).
$$
Прямоугольная область размером $m x n$, на которой задана импульсная характеристика, называется ядром фильтра.
\par
Имульсной характеристикой фильтра Гаусса является функция Гаусса. В одномерном случае она рассчитывается по формуле:
$$h(x) = \frac{1}{\sigma \sqrt{2 \pi}} e^{-\frac{(x - \mu)^2}{2 \sigma^2}}$$
Где $\sigma$ - среднеквадратичное отклонение, $\mu$ - математическое ожидание.
\par
Фильтр Гаусса является сепарабельным, поэтому для обработки двумерного сигнала (изображения) можно записать $h(i, j) = h(i) h(j)$. Интеграл от плотности вероятности для распределения Гаусса на интервале $[-3\sigma, 3\sigma]$, д.б. раен единице, поэтому коэффициенты вычисленной матрицы ыильтра (ядра) будет необходимо нормировать, поделив на их сумму.

\newpage

% Описание схемы распараллеливания
\subsection*{Описание схемы распараллеливания}
\addcontentsline{toc}{subsection}{Описание схемы распараллеливания}

\par
Работа алгоритма размытия состоит из двух частей: подсчет коэффициентов ядра фильтра и применение фильтра к изображению. Так как ядро фильтра имеет небольшие размеры (3 x 3), на его вычисление будет затрачено много меньше времени, чем на обход изображения. Поэтому имеет смысл писать параллельную версию именно для кода применения фильтра.
\par
Применение фильтра Гаусса заключается в применении операции свертки (применении ядра фильтра) к каждому пикселю изображения. Очевидным решением для оптимизации работы програмы будет распределение вычислений между потоками процессора следующим образом: каждый поток обрабатывает свой участок изображения (свой набор пикселов).
\par
Следуя технической задаче, организуем распределение данных между потоками по столбцам. Например, если программа запускается для $num = 4$ потоков и изображения $N x N$, то первый поток обрабатывает первые $N \% 4$ столбцов, второй - следующие $N \% 4$ и т.д. Если число столбцов не делится нацело на число потоков, то "лишние" столбцы достанутся последнему потоку.
\newpage

% Программная реализация
\subsection*{Программная реализация}
\addcontentsline{toc}{subsection}{Программная реализация}

\par
Последовательная реализация вычисления ядра фильтра Гаусса и применения фильтра к изображению находится в заголовочном файле \emph{include/gauss.h} и файле исходного кода \emph{src/gauss.cpp}. Функциональные тесты находятся в \emph{func\_tests/main.cpp}. Тесты производительности производятся в \emph{perf\_tests/main.cpp}.
\\
\textbf{Краткое описание структур данных:}
\\
\textbf{Color}
\begin{lstlisting}
struct Color {
  uint8_t R, G, B;
};
\end{lstlisting}
Представляет собой цвет в RGB-модели.
\\
\begin{itemize}
    \item R -- компонента красного цвета
    \item G -- компонента зеленого цвета
    \item B -- компонента синего цвета
\end{itemize}.
\\
\textbf{Изображение}
\begin{lstlisting}
  vector<Color> image;
\end{lstlisting}
Изображение храним в виде линейного массива - набора цветов пикселей.
\\
\textbf{Используемые переменные:}
\\
\begin{lstlisting}
  uint8_t *input, *output;
  uint32_t width, height;
  vector<Color> image;

  static const size_t kernelSize = 3;
  double kernel[3][3] = {{0, 0, 0}, {0, 0, 0}, {0, 0, 0}};
\end{lstlisting}
\textbf{Функция createKernel}
\begin{lstlisting}
void createKernel(float sigma = 2)
\end{lstlisting}
Вычисление ядра 3x3 фильтра Гаусса с заданным среднеквадратичным отклонением.
\\Входные данные:
\begin{itemize}
    \item sigma -- среднеквадратичное отклонение
\end{itemize}.
\\
\textbf{Функция calculateNewPixelColor} 
\begin{lstlisting}
Color calculateNewPixelColor(size_t x, size_t y)
\end{lstlisting}
Вычисляет цвет данного пикселя после применения фильтра.
\\Входные данные:
\begin{itemize}
    \item x -- координата по горизонталм
    \item y -- координата по вертикали
\end{itemize}
Возвращает: итоговый цвет пикселя.

\newpage
\section*{Тестирование и проведение экспериментов}
\addcontentsline{toc}{section}{Тестирование и проведение экспериментов}
% Подтверждение корректности
\subsection*{Подтверждение корректности}
\addcontentsline{toc}{subsection}{Подтверждение корректности}
\par Для подтверждения корректности работы программы были написаны тесты с использованием Google Test. Пять тестов проверяют корректность последовательной реализации метода на простой задаче, решение которой легко найти аналитически. Работа параллельных алгоритмов проверяетсся на тех же самых тестах и их результат сравнивается с результатом работы последовательной версии.
\par
Рассмотрим пример такого теста. Пусть изображение представляест из себя "шахматную доску" с значениями каждого пикселя $RGB(129, 129, 129)$ или $RGB(127, 127, 127)$. Нетрудно проверить, что при применении размытия к такому изображению итоговое изоюражение будет являться однотонным с цветом $RGB(128, 128, 128)$.
\newpage
% Результаты экспериментов
\subsection*{Результаты экспериментов}
\addcontentsline{toc}{subsection}{Результаты экспериментов}
Для проведения экспериментов по вычислению эффективности работы разных реализаций программы использовалась система со следующей конфигурацией:
\begin{itemize}
\item Процессор: Intel Core i3-10105F, ядер: 4, потоков: 8;
\item Оперативная память: 16 ГБ
\item Операционная система: Ubuntu 22.04.
\end{itemize}
Эксперимент ставился следующим образом: создавалось изображение фиксированного размера, . Для каждого эксперимента создавалось 4 процесса. Измерялось время выполнения последовательного алгоритма, параллельного алгоритма, а затем вычислялось ускорение (как отношение времени работы параллельного алгоритма к времени последовательного).
\par
Чтобы убедиться, что предобработка и постобработка изображения (в них входит получение и передача изображения, а также вычиселние ядра фильтра) занимают много меньше времени, чем основные вычисления, замеры проводились в двух видах: время выполнения основных вычислений и общее время работы алгоритма.
\par Результаты экспериментов:
\begin{table}[!h]
\centering
\begin{tabular}{| c | c | c | c |}
\hline
Версия программы & Время работы (с) & Ускорение (раз) & Эффективность\\
\hline
Последовательная  & 0.4203 & 1 & 0.25 \\
OpenMP  & 0.1436& 2.9272& 0.7318 \\
TBB  & 0.1286 & 3.2659 & 0.8164 \\
std::threads  & 0.1291& 3.2552 & 0.8138 \\
\hline
\end{tabular}
\caption{Результаты экспериментов: время работы всего алгоритма для изображения размером 1000x1000}
\end{table}
\begin{table}[!h]
\centering
\begin{tabular}{| c | c | c | c |}
\hline
Версия программы & Время работы (мс) & Ускорение (раз) & Эффективность\\
\hline
Последовательная  & 0.4093 & 1 & 0.25\\
OpenMP  & 0.1120 & 3.6540 & 0.9135\\
TBB  & 0.1059 & 3.8646 & 0.9661 \\
std::threads  & 0.1067 & 3.8373 & 0.9593 \\
\hline
\end{tabular}
\caption{Результаты экспериментов: время работы основной части алгоритма для изображения размером 1000x1000}
\end{table}
\par
Результаты эксперимента показывают, что схема распараллеливания была выбрана эффективно. Все три параллельные версии показывают сходные результаты, причем ускорение при $n=4$ потоках приближается к $4.0$, а эффективность близка к $1.0$.
\par
Также заметим, что время исполнения всей программы отличается от времени исполнения основных вычислений незначительно. Это говорит о том, что схема вычислений была организована корректно.
\newpage

% Заключение
\section*{Заключение}
\addcontentsline{toc}{section}{Заключение}
Таким образом, в рамках данной лабораторной работы был разработан алгоритм применения фильтра Гаусса для обработки изображения с использованием языка программирования C++. Для оптимизации вычислений были реализованы параллельные версии алгоритма с использованием технологий OpenMP, TBB, std::threads. Проведенные функциональные тесты показали корректность реализованной программы. Проведенные эксперименты доказали эффективность распараллеливания этого алгоритма. Причем все три технологии параллельного программирования показали сходные результаты.
\newpage

% Литература
\section*{Литература}
\addcontentsline{toc}{section}{Литература}
\begin{enumerate}
\item А.В. Сысоев, И.Б. Мееров, А.А. Сиднев «Средства разработки параллельных программ для систем с общей памятью. Библиотека Intel Threading Building Blocks». Нижний Новгород, 2007, 128 с. 
\item А.В. Сысоев, И.Б. Мееров, А.Н. Свистунов, А.Л. Курылев, А.В. Сенин, А.В. Шишков, К.В. Корняков, А.А. Сиднев «Параллельное программирование в системах с общей
памятью. Инструментальная поддержка». Нижний Новгород, 2007, 110 с. 
\item Корняков К. В. и др. Инструменты параллельного программирования в системах с общей. – 2010.
\item Фисенко В.Т., Фисенко Т.Ю. Компьютерная обработка и распознавание изображений/ Уч.пособие ИТМО. -2008. -192с.
\item А.Бовырин, П.Дружков, В.Ерухимов, Н.Золотых, В.Кустикова, И.Лысенков, И.Мееров, В.Писаревский, А.Половинкин, А.Сысоев. Академия Intel: Разработка мультимедийных приложений с использованием библиотек OpenCV и IPP. 
\item Gary Bradski and Adrian Kaehler. Learning OpenCV/ Published by O’Reilly Media, Inc., 2008. -577pp.
\item Курсы и материалы лаборатории Graphics Media Lab при ВМиК МГУ (http://graphics.cs.msu.ru) 

\end{enumerate}

\section*{Приложение}
\addcontentsline{toc}{section}{Приложение}
\subsection*{\emph{Ссылка на репозиторий с кодом:}}
\textit{\href{https://github.com/aaldebarann/ppc-2024-threads/blob/ivanchenko_a_gauss_filter_vertical}{https://github.com/aaldebarann/ppc-2024-threads/blob/ivanchenko\_a\_gauss\_filter\_vertical}}
\subsection*{\emph{include/gauss.h}}
\begin{lstlisting}
struct Color {
  uint8_t R, G, B;
};

class GaussFilterSequential : public ppc::core::Task {
 public:
  explicit GaussFilterSequential(std::shared_ptr<ppc::core::TaskData> taskData_) : Task(std::move(taskData_)) {}
  bool pre_processing() override;
  bool validation() override;
  bool run() override;
  bool post_processing() override;

 private:
  uint8_t *input, *output;
  uint32_t width, height;
  vector<Color> image;

  static const size_t kernelSize = 3;
  double kernel[3][3] = {{0, 0, 0}, {0, 0, 0}, {0, 0, 0}};
  void createKernel(float sigma = 2);
  void applyKernel();
  Color calculateNewPixelColor(size_t x, size_t y);
};
\end{lstlisting}

\subsection*{\emph{src/gauss.cpp}}
\begin{lstlisting}
void GaussFilterSequential::createKernel(float sigma) {
  uint32_t radius = kernelSize / 2;
  uint32_t size = kernelSize;
  float norm = 0;
  for (int i = -(int)radius; i <= (int)radius; i++) {
    for (int j = -(int)radius; j <= (int)radius; j++) {
      kernel[i + radius][j + radius] = (double)(std::exp(-(i * i + j * j) / (2 * sigma * sigma)));
      norm += kernel[i + radius][j + radius];
    }
  }
  for (uint32_t i = 0; i < size; i++) {
    for (uint32_t j = 0; j < size; j++) {
      kernel[i][j] /= norm;
    }
  }
}
void GaussFilterSequential::applyKernel() {
  for (uint32_t j = 0; j < height; j++) {
    for (uint32_t i = 0; i < width; i++) {
      image[i * width + j] = calculateNewPixelColor(i, j);
    }
  }
}
Color GaussFilterSequential::calculateNewPixelColor(size_t x, size_t y) {
  uint32_t radius = kernelSize / 2;
  float resultR = 0;
  float resultG = 0;
  float resultB = 0;
  for (int l = -((int)radius); l <= (int)radius; l++) {
    for (int k = -((int)radius); k <= (int)radius; k++) {
      size_t idX = x + k;
      size_t idY = y + l;
      if (idX < 0) idX = 0;
      if (idX >= width) idX = width - 1;
      if (idY < 0) idY = 0;
      if (idY >= height) idY = height - 1;
      Color neighborColor = image[idX * width + idY];
      resultR += neighborColor.R * kernel[k + radius][l + radius];
      resultG += neighborColor.G * kernel[k + radius][l + radius];
      resultB += neighborColor.B * kernel[k + radius][l + radius];
    }
  }
  Color result;
  result.R = (uint8_t)std::ceil(resultR);
  result.G = (uint8_t)std::ceil(resultG);
  result.B = (uint8_t)std::ceil(resultB);
  return result;
}
\end{lstlisting}

\end{document}
