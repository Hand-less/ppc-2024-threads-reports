\documentclass{report}

\usepackage[warn]{mathtext}
\usepackage[T2A]{fontenc}
\usepackage[utf8]{luainputenc}
\usepackage[english, russian]{babel}
\usepackage[pdfpagemode=UseNone,colorlinks,allcolors=black]{hyperref}
\usepackage{tempora}
\usepackage[12pt]{extsizes}
\usepackage{listings}
\usepackage{color}
\usepackage{geometry}
\usepackage{enumitem}
\usepackage{multirow}
\usepackage{graphicx}
\usepackage{indentfirst}
\usepackage{amsmath}
\usepackage{soul}

\sethlcolor{gray}

\geometry{a4paper,top=2cm,bottom=2cm,left=2.5cm,right=1.5cm}
\setlength{\parskip}{0.5cm}
\setlist{nolistsep, itemsep=0.3cm,parsep=0pt}

\usepackage{listings}
\lstset{language=C++,
        basicstyle=\footnotesize,
		keywordstyle=\color{blue}\ttfamily,
		stringstyle=\color{red}\ttfamily,
		commentstyle=\color{green}\ttfamily,
		morecomment=[l][\color{red}]{\#}, 
		tabsize=4,
		breaklines=true,
  		breakatwhitespace=true,
  		title=\lstname,       
}

\makeatletter
\renewcommand\@biblabel[1]{#1.\hfil}
\makeatother

\begin{document}

\begin{titlepage}

\begin{center}
Министерство науки и высшего образования Российской Федерации
\end{center}

\begin{center}
Федеральное государственное автономное образовательное учреждение высшего образования \\
Национальный исследовательский Нижегородский государственный университет \newline им. Н.И. Лобачевского
\end{center}

\begin{center}
Институт информационных технологий, математики и механики \\
Кафедра математического обеспечения и суперкомпьютерных технологий \\
Направление подготовки: Программная инженерия
\end{center}

\begin{center}
\textbf{Отчет по курсу \\
\vspace{0.5em}
<<Параллельное программирование для систем с общей памятью>>} \\
\end{center}

\vspace{4em}

\begin{center}
\textbf{\LargeТема \\
\vspace{0.5em}
<<Линейная фильтрация изображений (горизонтальное разбиение). Ядро Гаусса 3x3>>} \\
\end{center}

\vspace{4em}

\newbox{\lbox}
\savebox{\lbox}{\hbox{text}}
\newlength{\maxl}
\setlength{\maxl}{\wd\lbox}
\hfill\parbox{7cm}{
\hspace*{5cm}\hspace*{-5cm}\textbf{Выполнил:} \\ студент группы 3821Б1ПР2\\Филатов М. С.\\
\\
\hspace*{5cm}\hspace*{-5cm}\textbf{Проверил:}\\ аспирант\\Нестеров А. Ю.\\
}
\vspace{\fill}

\begin{center} Нижний Новгород \\ 2024 \end{center}

\end{titlepage}

\setcounter{page}{2}

% Содержание
\tableofcontents
\newpage

% Введение
\section*{Введение}
\addcontentsline{toc}{section}{Введение}
\par В эпоху стремительного развития технологий обработки цифровых изображений одной из ключевых задач является эффективное применение параллельных вычислений для повышения производительности алгоритмов. Линейная фильтрация, в частности с использованием ядра Гаусса размера 3x3, является распространенным методом улучшения качества изображений, устранения шумов и выделения важных деталей. Однако при обработке крупных изображений данный процесс может быть весьма ресурсоемким и требовать значительных вычислительных мощностей.

\par Целью настоящей работы является разработка и анализ эффективности параллельного алгоритма линейной фильтрации изображений с применением ядра Гаусса 3x3 и горизонтального разбиения изображения на системах с общей памятью. Горизонтальное разбиение позволяет распределить вычислительную нагрузку между несколькими потоками или процессами, что может существенно ускорить процесс фильтрации и повысить производительность.

\par В рамках данной работы будут рассмотрены теоретические основы линейной фильтрации изображений, описан параллельный алгоритм фильтрации с использованием ядра Гаусса и горизонтального разбиения, а также проведены эксперименты по оценке эффективности и ускорения предложенного алгоритма на различных наборах данных и конфигурациях системы. Результаты исследования могут быть полезны для разработчиков программного обеспечения в области обработки изображений и компьютерного зрения, а также для исследователей, занимающихся параллельными вычислениями и оптимизацией алгоритмов.

\par В рамках задачи реализован алгоритм с использованием технологий: SEQ, OMP, TBB, STD.

\newpage

% Постановка задачи
\section*{Постановка задачи}
\addcontentsline{toc}{section}{Постановка задачи}
\par \textbf{В рамках данного задания необходимо выполнить следующие задачи:}
\begin{enumerate}
\item Реализовать четыре версии алгоритма линейной фильтрации изображений с использованием ядра Гаусса 3x3 и горизонтального разбиения:
\begin{itemize}
\item Последовательная реализация (SEQ)
\item Параллельная реализация с использованием технологии OMP (OpenMP)
\item Параллельная реализация с использованием технологии TBB
\item Параллельная реализация с использованием стандартной библиотеки C++
\end{itemize}
\item Сравнить реализации и производительность версий.
\item Провести анализ полученных данных.
\end{enumerate}
\newpage

% Описание алгоритма
\section*{Описание алгоритма}
\addcontentsline{toc}{section}{Описание алгоритма}
\par Линейная фильтрация изображений с использованием ядра Гаусса 3x3 является широко применяемым методом для сглаживания изображений и удаления высокочастотных шумов. Данный алгоритм заключается в вычислении взвешенного среднего значения пикселей в окрестности каждого пикселя исходного изображения.

\par \textbf{Алгоритм линейной фильтрации с использованием ядра Гаусса 3x3 и горизонтального разбиения:}

\begin{enumerate}
\item Разделить исходное изображение на горизонтальные полосы.
\item Для каждой полосы выполнить следующие действия:
\begin{enumerate}
\item Применить ядро Гаусса 3x3 к каждому пикселю полосы, вычисляя взвешенное среднее значение соседних пикселей.
\item Обработать граничные пиксели, используя соответствующие значения из соседних полос.
\end{enumerate}
\item Объединить обработанные полосы в результирующее изображение.
\end{enumerate}

\par Горизонтальное разбиение изображения позволяет распределить вычислительную нагрузку между несколькими потоками или процессами, что может значительно ускорить процесс фильтрации и повысить производительность.

\par В рамках данной работы будут реализованы четыре версии алгоритма линейной фильтрации с использованием ядра Гаусса 3x3 и горизонтального разбиения:
\begin{itemize}
\item Последовательная реализация (SEQ)
\item Параллельная реализация с использованием технологии (OMP) OpenMP
\item Параллельная реализация с использованием технологии TBB
\item Параллельная реализация с использованием стандартной библиотеки C++
\end{itemize}
Будет проведено сравнение реализаций и анализ производительности каждой версии.

\newpage

% Описание схемы распараллеливания
\section* {Описание схемы распараллеливания}
\addcontentsline{toc}{section}{Описание схемы распараллеливания}
\par \textbf{Алгоритм линейной фильтрации изображений с использованием ядра Гаусса 3x3 и горизонтального разбиения состоит из следующих основных этапов, которые были распараллелены:}
\begin{enumerate}
\item Копирование входных данных в изображение
\item Применение ядра Гаусса 3x3 к каждому пикселю изображения
\item Копирование обработанного изображения в выходные данные
\end{enumerate}
\vspace{2em}
\par \textbf{На втором этапе (применение ядра Гаусса 3x3) было распараллелено:}
\begin{enumerate}
\item Вычисление новых значений цветовых компонент каждого пикселя с использованием ядра Гаусса 3x3
\item Ограничение значений цветовых компонент до допустимого диапазона
\end{enumerate}
\vspace{1em}
\par Горизонтальное разбиение изображения позволяет распределить вычислительную нагрузку между несколькими потоками или процессами, что может значительно ускорить процесс фильтрации и повысить производительность. В реализации OMP использовалась директива \texttt pragma omp parallel for для распараллеливания цикла, выполняющего применение ядра Гаусса 3x3 к каждому пикселю изображения. В других реализациях с использованием других технолигий распараллеливалась та же секция.

\newpage

% Описание OpenMP, TBB и STL версий
\section* {Описание OpenMP, TBB и STL версий}
\par В различных версиях алгоритма линейной фильтрации изображений с использованием ядра Гаусса 3x3 и горизонтального разбиения, использующих параллельные вычисления, основные участки кода, подлежащие распараллеливанию, остаются неизменными. Однако ключевое различие между версиями, такими как OpenMP, TBB (Threading Building Blocks) и STL (Standard Template Library), заключается в предоставляемых ими возможностях и методах распараллеливания.
\par \textbf{Версии реализации:}
\begin{enumerate}
\item В OpenMP были использованы директивы:
\vspace{0.5em}
\begin{enumerate}
\item[1.1] \begin{verbatim}
#pragma omp parallel for
\end{verbatim}

для распараллеливания цикла, выполняющего применение ядра Гаусса 3x3 к каждому пикселю изображения
\end{enumerate}
\item В TBB было использовано:
\vspace{0.5em}
\begin{enumerate}
\item[2.1] \begin{verbatim}
tbb::parallel_for(tbb::blocked_range2d(0, height, 0, width), [&](const tbb::blocked_range2d& r) { ... })
\end{verbatim}
для распараллеливания вычисления новых значений цветовых компонент для каждого пикселя изображения
\end{enumerate}
\item В STL было использовано:
\vspace{0.5em}
\begin{enumerate}
\item[3.1] \begin{verbatim}
std::thread
\end{verbatim}
для создания потоков, выполняющих применение ядра Гаусса 3x3 к различным частям изображения
\item[3.2] \begin{verbatim}
std::thread::join
\end{verbatim}
для ожидания завершения выполнения всех потоков
\end{enumerate}
\end{enumerate}

Во всех трех версиях основными распараллеливаемыми участками кода являются:
\begin{enumerate}
\item Вычисление новых значений цветовых компонент для каждого пикселя изображения с использованием ядра Гаусса 3x3
\item Ограничение значений цветовых компонент до допустимого диапазона
\end{enumerate}

Различия между версиями заключаются в используемых средствах параллельного программирования и способах распределения вычислительной нагрузки между потоками.

\newpage

% Результаты экспериментов
\section*{Результаты экспериментов}
\addcontentsline{toc}{section}{Результаты экспериментов}
\par Корректность алгоритма линейной фильтрации изображений с использованием ядра Гаусса 3x3 и горизонтального разбиения была проверена путем сравнения результатов фильтрации с эталонными изображениями.
\par Тестирование производительности проводилось на изображениях различных размеров, начиная от 512x512 пикселей и до 2048x2048 пикселей.
\par Для проведения экспериментов по вычислению эффективности работы алгоритмов использовалась система со следующей конфигурацией:
\begin{itemize}
\item Процессор: AMD Ryzen 7 4800H, 8 ядер / 16 потоков
\item Оперативная память: 16 ГБ
\item Операционная система: Windows 10
\end{itemize}
\par Количество потоков для параллельных версий алгоритма было ограничено до 4.
\par В таблице приведены средние значения времени обработки изображения размером 1024х1024 пикселей за 10 запусков.
\par \textbf{Результаты экспериментов:}
\begin{center}
\begin{tabular}{ ||c | c | c ||  }
    \hline Версия алгоритма & pipeline (сек) & task (сек)\\ 
    \hline Последовательная & 2.1235411100 & 2.0725412400 \\
    \hline OMP & 1.1183264209 & 1.0781151349 \\
    \hline TBB & 1.0841456712 & 1.0515942547 \\ 
    \hline STL & 1.1245168910 & 1.0956216933 \\ 
    \hline
\end{tabular}\\[5mm]
\end{center}

\newpage

% Анализ результатов
\section*{Анализ результатов}
\addcontentsline{toc}{section}{Анализ результатов}
\par Для оценки эффективности параллельных алгоритмов линейной фильтрации изображений с использованием ядра Гаусса 3x3, были рассчитаны коэффициенты ускорения и эффективности. Расчеты основаны на среднем времени выполнения алгоритмов для изображения размером 1024х1024 пикселей.

\par Коэффициент ускорения \( p_i \) для каждой версии алгоритма рассчитывается по формуле:
$$ p_i = \frac{T_{si}}{T_{pi}} $$
где \( T_{si} \) - время выполнения последовательного алгоритма, а \( T_{pi} \) - время выполнения параллельного алгоритма для \( i \)-й версии.

\par Эффективность \( e_i \) для каждой версии алгоритма рассчитывается по формуле:
$$ e_i = \frac{p_i}{N} $$
где \( N \) - количество потоков (в данном случае 4).

\par Используя данные из предыдущей таблицы, получаем следующие результаты для двух различных типов задач: pipeline и task.

\begin{center}
\begin{tabular}{ ||c | c | c ||}
 \hline
 \multicolumn{3}{| c |}{pipeline}\\
 \hline
 Версия & Ускорение & Эффективность\\
 \hline
 Последовательная & 1 & 0.25 \\
 OpenMP & \( \frac{2.1235411100}{1.1183264209} \approx 1.90 \) & \( \frac{1.90}{4} \approx 0.475 \) \\
 TBB & \( \frac{2.1235411100}{1.0841456712} \approx 1.96 \) & \( \frac{1.96}{4} \approx 0.49 \) \\
 STL & \( \frac{2.1235411100}{1.1245168910} \approx 1.89 \) & \( \frac{1.89}{4} \approx 0.472 \) \\
 \hline
\end{tabular}
\end{center}

\vspace{2em}

\begin{center}
\begin{tabular}{ ||c | c | c ||}
 \hline
 \multicolumn{3}{| c |}{task}\\
 \hline
 Версия & Ускорение & Эффективность\\
 \hline
 Последовательная & 1 & 0.25 \\
 OpenMP & \( \frac{2.0725412400}{1.0781151349} \approx 1.92 \) & \( \frac{1.92}{4} \approx 0.48 \) \\
 TBB & \( \frac{2.0725412400}{1.0515942547} \approx 1.97 \) & \( \frac{1.97}{4} \approx 0.492 \) \\
 STL & \( \frac{2.0725412400}{1.0956216933} \approx 1.89 \) & \( \frac{1.89}{4} \approx 0.472 \) \\
 \hline
\end{tabular}
\end{center}

\textbf{Выводы}:
\begin{itemize}
    \item В обоих типах задач (pipeline и task) наилучшее ускорение и эффективность показала версия TBB.
    \item Наименьшее ускорение и эффективность демонстрирует STL версия.
    \item Распараллеливание алгоритма привело к ускорению его выполнения примерно в два раза по сравнению с последовательной версией.
\end{itemize}

\newpage

% Заключение
\section*{Заключение}
\addcontentsline{toc}{section}{Заключение}
\par В ходе данной лабораторной работы были успешно реализованы и проанализированы как последовательный, так и параллельные алгоритмы линейной фильтрации изображений с использованием ядра Гаусса 3x3. Экспериментальные замеры производительности подтвердили, что распараллеливание алгоритма значительно увеличивает его эффективность, особенно на системах с многопоточной обработкой.

\par Оценка удобства использования различных технологий параллелизма показала, что, несмотря на некоторые накладные расходы, TBB и OpenMP предоставляют мощные инструменты для оптимизации производительности. STL, хотя и оказался менее эффективным в данном контексте, все же предоставляет ценные возможности для упрощения многопоточного программирования.

\par Ранжирование технологий по удобству использования, на основе личного опыта и результатов тестирования, выглядит следующим образом:
\begin{itemize}
    \item[1.] TBB — за его высокую производительность и удобство в управлении потоками.
    \item[2.] OpenMP — за простоту внедрения и хорошую поддержку компиляторами.
    \item[3.] STL — за его доступность и интеграцию с стандартной библиотекой C++.
\end{itemize}

\par В заключение, результаты данной работы могут служить отправной точкой для дальнейших исследований в области параллельных вычислений и оптимизации алгоритмов обработки изображений.

\newpage


% Литература
\section*{Литература}
\addcontentsline{toc}{section}{Литература}
\begin{enumerate}
    \item Лекции Сысоева А.В. по курсу "Параллельное программирование для систем с общей памятью".
    \item А.В. Сысоев, И.Б. Мееров, А.Н. Свистунов, А.Л. Курылев, А.В. Сенин, А.В. Шишков, К.В. Корняков, А.А. Сиднев «Параллельное программирование в системах с общей памятью. Инструментальная поддержка». Нижний Новгород, 2007, 110 с.
    \item Матричные фильтры обработки изображений. Доступно: \url{https://habr.com/ru/articles/142818/}
    \item Основы обработки изображений. Доступно: \url{https://cmcmsu.info/download/cv2015_03_ip.pdf}
\end{enumerate}

\newpage

\section*{Приложение}
\addcontentsline{toc}{section}{Приложение}

\begin{lstlisting}[language=C++,caption=Последовательная версия]
// Copyright 2024 Filatov Maxim
#include "seq/filatov_m_linear_image_filtering/include/ops_seq.hpp"

#include <cstdint>

Color::Color() { R = G = B = 0; }

ColorF::ColorF() { R = G = B = .0f; }

void GaussFilterHorizontal::initializeData() {
  input = taskData->inputs[0];
  output = taskData->outputs[0];
  width = taskData->inputs_count[0];
  height = taskData->inputs_count[1];
}

void GaussFilterHorizontal::copyInputToImage() {
  image.resize(width * height * 3);
  std::memcpy(image.data(), input, width * height * 3);
}

bool GaussFilterHorizontal::pre_processing() {
  internal_order_test();
  initializeData();
  copyInputToImage();
  makeKernel();
  return true;
}

bool GaussFilterHorizontal::validation_is_input_size_valid() {
  return taskData->inputs_count[0] >= 3 && taskData->inputs_count[1] >= 3;
}

bool GaussFilterHorizontal::validation_is_output_size_valid_first() {
  return taskData->outputs_count[0] == taskData->inputs_count[0];
}

bool GaussFilterHorizontal::validation_is_output_size_valid_second() {
  return taskData->outputs_count[1] == taskData->inputs_count[1];
}

bool GaussFilterHorizontal::validation() {
  internal_order_test();
  bool isInputSizeValid = validation_is_input_size_valid();
  bool isOutputSizeValidFirst = validation_is_output_size_valid_first();
  bool isOutputSizeValidSecond = validation_is_output_size_valid_second();
  bool isOutputSizeValid = isOutputSizeValidFirst && isOutputSizeValidSecond;
  return isInputSizeValid && isOutputSizeValid;
}

bool GaussFilterHorizontal::run() {
  internal_order_test();
  applyKernel();
  return true;
}

bool GaussFilterHorizontal::copyImageData() {
  std::memcpy(output, image.data(), width * height * 3);
  return true;
}

bool GaussFilterHorizontal::post_processing() {
  internal_order_test();
  return copyImageData();
}

void GaussFilterHorizontal::calculateGaussianValues(float sigma, float* normalizationFactor) {
  int64_t halfSize = kSize * .5f;
  for (int64_t row = -halfSize; row <= halfSize; row++) {
    for (int64_t col = -halfSize; col <= halfSize; col++) {
      double gaussianValue = std::exp(-(row * row + col * col) / (2 * sigma * sigma));
      kernel[row + halfSize][col + halfSize] = gaussianValue;
      *normalizationFactor += gaussianValue;
    }
  }
}

void GaussFilterHorizontal::normalizeKernel(float normalizationFactor) {
  for (uint32_t row = 0; row < kSize; row++) {
    for (uint32_t col = 0; col < kSize; col++) {
      kernel[row][col] /= normalizationFactor;
    }
  }
}

void GaussFilterHorizontal::makeKernel(float sigma) {
  float normalizationFactor = 0;
  calculateGaussianValues(sigma, &normalizationFactor);
  normalizeKernel(normalizationFactor);
}

void GaussFilterHorizontal::applyKernel() {
  uint32_t index = 0;
  for (auto& pixel : image) {
    uint32_t i = index / width;
    uint32_t j = index % width;
    pixel = calculateNewPixelColor(i, j);
    index++;
  }
}

void GaussFilterHorizontal::calculateSingleColorComponent(uint8_t neighborColor, float kernelValue,
                                                          float* newColorComponent) {
  *newColorComponent += neighborColor * kernelValue;
}

void GaussFilterHorizontal::calculateColorsComponents(Color* neighborColor, int64_t k, int64_t l, int64_t halfSize,
                                                      ColorF* color) {
  calculateSingleColorComponent(neighborColor->R, kernel[k + halfSize][l + halfSize], &color->R);
  calculateSingleColorComponent(neighborColor->G, kernel[k + halfSize][l + halfSize], &color->G);
  calculateSingleColorComponent(neighborColor->B, kernel[k + halfSize][l + halfSize], &color->B);
}

void GaussFilterHorizontal::calculateColorComponentsForRow(int64_t l, size_t x, size_t y, ColorF* color) {
  int64_t halfSize = kSize * .5f;
  for (int64_t k = -halfSize; k <= halfSize; k++) {
    auto idX = clamp<size_t>((x + k), 0, width - 1);
    auto idY = clamp<size_t>((y + l), 0, height - 1);
    Color neighborColor = image[idX * width + idY];
    calculateColorsComponents(&neighborColor, k, l, halfSize, color);
  }
}

void GaussFilterHorizontal::calculateColorComponents(size_t x, size_t y, ColorF* color) {
  int64_t halfSize = kSize * .5f;
  for (int64_t l = -halfSize; l <= halfSize; l++) {
    calculateColorComponentsForRow(l, x, y, color);
  }
}

Color GaussFilterHorizontal::calculateCeilColorF(ColorF preparedColor) {
  Color resultColor;
  resultColor.R = (uint8_t)std::ceil(preparedColor.R);
  resultColor.G = (uint8_t)std::ceil(preparedColor.G);
  resultColor.B = (uint8_t)std::ceil(preparedColor.B);
  return resultColor;
}

Color GaussFilterHorizontal::calculateNewPixelColor(size_t x, size_t y) {
  ColorF preparedColor;
  calculateColorComponents(x, y, &preparedColor);
  return calculateCeilColorF(preparedColor);
}

template <typename T>
T GaussFilterHorizontal::clamp(const T& val, const T& min, const T& max) {
  if (val < min) {
    return min;
  }
  if (val > max) {
    return max;
  }
  return val;
}
\end{lstlisting}

\newpage

\begin{lstlisting}[language=C++,caption=OpenMP версия]
// Copyright 2024 Filatov Maxim
#include "omp/filatov_m_linear_image_filtering/include/ops_omp.hpp"

#include <cstdint>

filatov_omp::Color::Color() { R = G = B = 0; }

uint8_t filatov_omp::Color::convert(float var) { return static_cast<uint8_t>(std::ceil(var)); }

void filatov_omp::Color::setAll(ColorF cf) {
  R = convert(cf.R);
  G = convert(cf.G);
  B = convert(cf.B);
}

filatov_omp::ColorF::ColorF() { R = G = B = .0f; }

void filatov_omp::GaussFilterHorizontal::initializeData() {
  input = taskData->inputs[0];
  output = taskData->outputs[0];
  width = taskData->inputs_count[0];
  height = taskData->inputs_count[1];
}

void filatov_omp::GaussFilterHorizontal::copyInputToImage() {
  image.resize(width * height * 3);
  std::memcpy(image.data(), input, width * height * 3);
}

bool filatov_omp::GaussFilterHorizontal::pre_processing() {
  internal_order_test();
  initializeData();
  copyInputToImage();
  makeKernel();
  return true;
}

bool filatov_omp::GaussFilterHorizontal::validation_is_input_size_valid() {
  return taskData->inputs_count[0] >= 3 && taskData->inputs_count[1] >= 3;
}

bool filatov_omp::GaussFilterHorizontal::validation_is_output_size_valid_first() {
  return taskData->outputs_count[0] == taskData->inputs_count[0];
}

bool filatov_omp::GaussFilterHorizontal::validation_is_output_size_valid_second() {
  return taskData->outputs_count[1] == taskData->inputs_count[1];
}

bool filatov_omp::GaussFilterHorizontal::validation() {
  internal_order_test();
  bool isInputSizeValid = validation_is_input_size_valid();
  bool isOutputSizeValidFirst = validation_is_output_size_valid_first();
  bool isOutputSizeValidSecond = validation_is_output_size_valid_second();
  bool isOutputSizeValid = isOutputSizeValidFirst && isOutputSizeValidSecond;
  return isInputSizeValid && isOutputSizeValid;
}

bool filatov_omp::GaussFilterHorizontal::run() {
  internal_order_test();
  applyKernel();
  return true;
}

bool filatov_omp::GaussFilterHorizontal::copyImageData() {
  std::memcpy(output, image.data(), width * height * 3);
  return true;
}

bool filatov_omp::GaussFilterHorizontal::post_processing() {
  internal_order_test();
  return copyImageData();
}

void filatov_omp::GaussFilterHorizontal::calculateGaussianValues(float sigma, float* normalizationFactor) {
  int64_t halfSize = kSize * .5f;
  for (int64_t row = -halfSize; row <= halfSize; row++) {
    for (int64_t col = -halfSize; col <= halfSize; col++) {
      double gaussianValue = std::exp(-(row * row + col * col) / (2 * sigma * sigma));
      kernel[row + halfSize][col + halfSize] = gaussianValue;
      *normalizationFactor += gaussianValue;
    }
  }
}

void filatov_omp::GaussFilterHorizontal::normalizeKernel(float normalizationFactor) {
  for (uint32_t row = 0; row < kSize; row++) {
    for (uint32_t col = 0; col < kSize; col++) {
      kernel[row][col] /= normalizationFactor;
    }
  }
}

void filatov_omp::GaussFilterHorizontal::makeKernel(float sigma) {
  float normalizationFactor = 0;
  calculateGaussianValues(sigma, &normalizationFactor);
  normalizeKernel(normalizationFactor);
}

void filatov_omp::GaussFilterHorizontal::applyKernel() {
#pragma omp parallel for
  for (int32_t index = 0; index < static_cast<int32_t>(image.size()); index++) {
    int32_t i = index / width;
    int32_t j = index % width;
    image[index] = calculateNewPixelColor(i, j);
  }
}

void filatov_omp::GaussFilterHorizontal::calculateSingleColorComponent(uint8_t neighborColor, float kernelValue,
                                                                       float* newColorComponent) {
  *newColorComponent += neighborColor * kernelValue;
}

void filatov_omp::GaussFilterHorizontal::calculateColorsComponents(Color* neighborColor, int64_t k, int64_t l,
                                                                   int64_t halfSize, ColorF* color) {
  calculateSingleColorComponent(neighborColor->R, kernel[k + halfSize][l + halfSize], &color->R);
  calculateSingleColorComponent(neighborColor->G, kernel[k + halfSize][l + halfSize], &color->G);
  calculateSingleColorComponent(neighborColor->B, kernel[k + halfSize][l + halfSize], &color->B);
}

void filatov_omp::GaussFilterHorizontal::calculateColorComponentsForRow(int64_t l, size_t x, size_t y, ColorF* color) {
  int64_t halfSize = kSize * .5f;
  for (int64_t k = -halfSize; k <= halfSize; k++) {
    auto idX = clamp<size_t>((x + k), 0, width - 1);
    auto idY = clamp<size_t>((y + l), 0, height - 1);
    Color neighborColor = image[idX * width + idY];
    calculateColorsComponents(&neighborColor, k, l, halfSize, color);
  }
}

void filatov_omp::GaussFilterHorizontal::calculateColorComponents(size_t x, size_t y, ColorF* color) {
  int64_t halfSize = kSize * .5f;
  for (int64_t l = -halfSize; l <= halfSize; l++) {
    calculateColorComponentsForRow(l, x, y, color);
  }
}

filatov_omp::Color filatov_omp::GaussFilterHorizontal::calculateCeilColorF(ColorF preparedColor) {
  Color resultColor;
  resultColor.setAll(preparedColor);
  return resultColor;
}

filatov_omp::Color filatov_omp::GaussFilterHorizontal::calculateNewPixelColor(size_t x, size_t y) {
  ColorF preparedColor;
  calculateColorComponents(x, y, &preparedColor);
  return calculateCeilColorF(preparedColor);
}

template <typename T>
T filatov_omp::GaussFilterHorizontal::clamp(const T& val, const T& min, const T& max) {
  return (val < min ? min : (val > max ? max : val));
}
\end{lstlisting}

\newpage

\begin{lstlisting}[language=C++,caption=TBB версия]
// Copyright 2024 Filatov Maxim
#include "tbb/filatov_m_linear_image_filtering_tbb/include/ops_tbb.hpp"

#include <cstdint>

filatov_tbb::Color::Color() { R = G = B = 0; }

filatov_tbb::ColorF::ColorF() { R = G = B = .0f; }

void filatov_tbb::GaussFilterHorizontal::initializeData() {
  input = taskData->inputs[0];
  output = taskData->outputs[0];
  width = taskData->inputs_count[0];
  height = taskData->inputs_count[1];
}

void filatov_tbb::GaussFilterHorizontal::copyInputToImage() {
  image.resize(width * height * 3);
  std::memcpy(image.data(), input, width * height * 3);
}

bool filatov_tbb::GaussFilterHorizontal::pre_processing() {
  internal_order_test();
  initializeData();
  copyInputToImage();
  makeKernel();
  return true;
}

bool filatov_tbb::GaussFilterHorizontal::validation_is_input_size_valid() {
  return taskData->inputs_count[0] >= 3 && taskData->inputs_count[1] >= 3;
}

bool filatov_tbb::GaussFilterHorizontal::validation_is_output_size_valid_first() {
  return taskData->outputs_count[0] == taskData->inputs_count[0];
}

bool filatov_tbb::GaussFilterHorizontal::validation_is_output_size_valid_second() {
  return taskData->outputs_count[1] == taskData->inputs_count[1];
}

bool filatov_tbb::GaussFilterHorizontal::validation() {
  internal_order_test();
  bool isInputSizeValid = validation_is_input_size_valid();
  bool isOutputSizeValidFirst = validation_is_output_size_valid_first();
  bool isOutputSizeValidSecond = validation_is_output_size_valid_second();
  bool isOutputSizeValid = isOutputSizeValidFirst && isOutputSizeValidSecond;
  return isInputSizeValid && isOutputSizeValid;
}

bool filatov_tbb::GaussFilterHorizontal::run() {
  internal_order_test();
  applyKernel();
  return true;
}

bool filatov_tbb::GaussFilterHorizontal::copyImageData() {
  std::memcpy(output, image.data(), width * height * 3);
  return true;
}

bool filatov_tbb::GaussFilterHorizontal::post_processing() {
  internal_order_test();
  return copyImageData();
}

void filatov_tbb::GaussFilterHorizontal::calculateGaussianValues(float sigma, float* normalizationFactor) {
  int64_t halfSize = kSize * .5f;
  for (int64_t row = -halfSize; row <= halfSize; row++) {
    for (int64_t col = -halfSize; col <= halfSize; col++) {
      double gaussianValue = std::exp(-(row * row + col * col) / (2 * sigma * sigma));
      kernel[row + halfSize][col + halfSize] = gaussianValue;
      *normalizationFactor += gaussianValue;
    }
  }
}

void filatov_tbb::GaussFilterHorizontal::normalizeKernel(float normalizationFactor) {
  for (uint32_t row = 0; row < kSize; row++) {
    for (uint32_t col = 0; col < kSize; col++) {
      kernel[row][col] /= normalizationFactor;
    }
  }
}

void filatov_tbb::GaussFilterHorizontal::makeKernel(float sigma) {
  float normalizationFactor = 0;
  calculateGaussianValues(sigma, &normalizationFactor);
  normalizeKernel(normalizationFactor);
}
void filatov_tbb::GaussFilterHorizontal::applyKernel() {
  tbb::parallel_for(tbb::blocked_range2d<uint32_t>(0, height, 0, width), [&](const tbb::blocked_range2d<uint32_t>& r) {
    for (uint32_t i = r.rows().begin(); i != r.rows().end(); ++i) {
      for (uint32_t j = r.cols().begin(); j != r.cols().end(); ++j) {
        size_t index = i * width + j;
        image[index] = calculateNewPixelColor(i, j);
      }
    }
  });
}

void filatov_tbb::GaussFilterHorizontal::calculateSingleColorComponent(uint8_t neighborColor, float kernelValue,
                                                                       float* newColorComponent) {
  *newColorComponent += neighborColor * kernelValue;
}

void filatov_tbb::GaussFilterHorizontal::calculateColorsComponents(Color* neighborColor, int64_t k, int64_t l,
                                                                   int64_t halfSize, ColorF* color) {
  calculateSingleColorComponent(neighborColor->R, kernel[k + halfSize][l + halfSize], &color->R);
  calculateSingleColorComponent(neighborColor->G, kernel[k + halfSize][l + halfSize], &color->G);
  calculateSingleColorComponent(neighborColor->B, kernel[k + halfSize][l + halfSize], &color->B);
}

void filatov_tbb::GaussFilterHorizontal::calculateColorComponentsForRow(int64_t l, size_t x, size_t y, ColorF* color) {
  int64_t halfSize = kSize * .5f;
  for (int64_t k = -halfSize; k <= halfSize; k++) {
    auto idX = clamp<size_t>((x + k), 0, width - 1);
    auto idY = clamp<size_t>((y + l), 0, height - 1);
    Color neighborColor = image[idX * width + idY];
    calculateColorsComponents(&neighborColor, k, l, halfSize, color);
  }
}

void filatov_tbb::GaussFilterHorizontal::calculateColorComponents(size_t x, size_t y, ColorF* color) {
  int64_t halfSize = kSize * .5f;
  for (int64_t l = -halfSize; l <= halfSize; l++) {
    calculateColorComponentsForRow(l, x, y, color);
  }
}

filatov_tbb::Color filatov_tbb::GaussFilterHorizontal::calculateCeilColorF(ColorF preparedColor) {
  Color resultColor;
  resultColor.R = (uint8_t)std::ceil(preparedColor.R);
  resultColor.G = (uint8_t)std::ceil(preparedColor.G);
  resultColor.B = (uint8_t)std::ceil(preparedColor.B);
  return resultColor;
}

filatov_tbb::Color filatov_tbb::GaussFilterHorizontal::calculateNewPixelColor(size_t x, size_t y) {
  ColorF preparedColor;
  calculateColorComponents(x, y, &preparedColor);
  return calculateCeilColorF(preparedColor);
}

template <typename T>
T filatov_tbb::GaussFilterHorizontal::clamp(const T& val, const T& min, const T& max) {
  if (val < min) {
    return min;
  }
  if (val > max) {
    return max;
  }
  return val;
}
\end{lstlisting}

\newpage

\begin{lstlisting}[language=C++,caption=STL версия]
// Copyright 2024 Filatov Maxim
#include "stl/filatov_m_linear_image_filtering_stl/include/ops_stl.hpp"

#include <cstdint>

filatov_stl::Color::Color() { R = G = B = 0; }

filatov_stl::ColorF::ColorF() { R = G = B = .0f; }

void filatov_stl::GaussFilterHorizontal::initializeData() {
  input = taskData->inputs[0];
  output = taskData->outputs[0];
  width = taskData->inputs_count[0];
  height = taskData->inputs_count[1];
}

void filatov_stl::GaussFilterHorizontal::copyInputToImage() {
  image.resize(width * height * 3);
  std::memcpy(image.data(), input, width * height * 3);
}

bool filatov_stl::GaussFilterHorizontal::pre_processing() {
  internal_order_test();
  initializeData();
  copyInputToImage();
  makeKernel();
  return true;
}

bool filatov_stl::GaussFilterHorizontal::validation_is_input_size_valid() {
  return taskData->inputs_count[0] >= 3 && taskData->inputs_count[1] >= 3;
}

bool filatov_stl::GaussFilterHorizontal::validation_is_output_size_valid_first() {
  return taskData->outputs_count[0] == taskData->inputs_count[0];
}

bool filatov_stl::GaussFilterHorizontal::validation_is_output_size_valid_second() {
  return taskData->outputs_count[1] == taskData->inputs_count[1];
}

bool filatov_stl::GaussFilterHorizontal::validation() {
  internal_order_test();
  bool isInputSizeValid = validation_is_input_size_valid();
  if (!isInputSizeValid) {
    std::cerr << "Error: incorrect input data size." << std::endl;
  }

  bool isOutputSizeValidFirst = validation_is_output_size_valid_first();
  if (!isOutputSizeValidFirst) {
    std::cerr << "Error: incorrect output data size (first parameter)." << std::endl;
  }

  bool isOutputSizeValidSecond = validation_is_output_size_valid_second();
  if (!isOutputSizeValidSecond) {
    std::cerr << "Error: incorrect output data size (second parameter)." << std::endl;
  }

  bool isOutputSizeValid = isOutputSizeValidFirst && isOutputSizeValidSecond;
  return isInputSizeValid && isOutputSizeValid;
}

bool filatov_stl::GaussFilterHorizontal::run() {
  internal_order_test();
  applyKernel();
  return true;
}

bool filatov_stl::GaussFilterHorizontal::copyImageData() {
  std::memcpy(output, image.data(), width * height * 3);
  return true;
}

bool filatov_stl::GaussFilterHorizontal::post_processing() {
  internal_order_test();
  return copyImageData();
}

void filatov_stl::GaussFilterHorizontal::calculateGaussianValues(float sigma, float* normalizationFactor) {
  int64_t halfSize = kSize * .5f;
  for (int64_t row = -halfSize; row <= halfSize; row++) {
    for (int64_t col = -halfSize; col <= halfSize; col++) {
      double gaussianValue = std::exp(-(row * row + col * col) / (2 * sigma * sigma));
      kernel[row + halfSize][col + halfSize] = gaussianValue;
      *normalizationFactor += gaussianValue;
    }
  }
}

void filatov_stl::GaussFilterHorizontal::normalizeKernel(float normalizationFactor) {
  for (uint32_t row = 0; row < kSize; row++) {
    for (uint32_t col = 0; col < kSize; col++) {
      kernel[row][col] /= normalizationFactor;
    }
  }
}

void filatov_stl::GaussFilterHorizontal::makeKernel(float sigma) {
  float normalizationFactor = 0;
  calculateGaussianValues(sigma, &normalizationFactor);
  normalizeKernel(normalizationFactor);
}

void filatov_stl::GaussFilterHorizontal::applyKernel() {
  size_t num_threads = std::thread::hardware_concurrency();
  size_t part_size = image.size() / num_threads;
  std::vector<std::thread> threads;

  for (size_t i = 0; i < num_threads; ++i) {
    size_t start = i * part_size;
    size_t end = (i == num_threads - 1) ? image.size() : (i + 1) * part_size;
    threads.emplace_back(&GaussFilterHorizontal::applyKernelPart, this, start, end);
  }

  for (auto& t : threads) {
    t.join();
  }
}

void filatov_stl::GaussFilterHorizontal::applyKernelPart(size_t start, size_t end) {
  for (size_t index = start; index < end; ++index) {
    uint32_t i = index / width;
    uint32_t j = index % width;
    image[index] = calculateNewPixelColor(i, j);
  }
}

void filatov_stl::GaussFilterHorizontal::calculateSingleColorComponent(uint8_t neighborColor, float kernelValue,
                                                                       float* newColorComponent) {
  *newColorComponent += neighborColor * kernelValue;
}

void filatov_stl::GaussFilterHorizontal::calculateColorsComponents(Color* neighborColor, int64_t k, int64_t l,
                                                                   int64_t halfSize, ColorF* color) {
  calculateSingleColorComponent(neighborColor->R, kernel[k + halfSize][l + halfSize], &color->R);
  calculateSingleColorComponent(neighborColor->G, kernel[k + halfSize][l + halfSize], &color->G);
  calculateSingleColorComponent(neighborColor->B, kernel[k + halfSize][l + halfSize], &color->B);
}

void filatov_stl::GaussFilterHorizontal::calculateColorComponentsForRow(int64_t l, size_t x, size_t y, ColorF* color) {
  int64_t halfSize = kSize * .5f;
  for (int64_t k = -halfSize; k <= halfSize; k++) {
    auto idX = clamp<size_t>((x + k), 0, width - 1);
    auto idY = clamp<size_t>((y + l), 0, height - 1);
    Color neighborColor = image[idX * width + idY];
    calculateColorsComponents(&neighborColor, k, l, halfSize, color);
  }
}

void filatov_stl::GaussFilterHorizontal::calculateColorComponents(size_t x, size_t y, ColorF* color) {
  int64_t halfSize = kSize * .5f;
  for (int64_t l = -halfSize; l <= halfSize; l++) {
    calculateColorComponentsForRow(l, x, y, color);
  }
}

filatov_stl::Color filatov_stl::GaussFilterHorizontal::calculateCeilColorF(ColorF preparedColor) {
  Color resultColor;
  resultColor.R = (uint8_t)std::ceil(preparedColor.R);
  resultColor.G = (uint8_t)std::ceil(preparedColor.G);
  resultColor.B = (uint8_t)std::ceil(preparedColor.B);
  return resultColor;
}

filatov_stl::Color filatov_stl::GaussFilterHorizontal::calculateNewPixelColor(size_t x, size_t y) {
  ColorF preparedColor;
  calculateColorComponents(x, y, &preparedColor);
  return calculateCeilColorF(preparedColor);
}

template <typename T>
T filatov_stl::GaussFilterHorizontal::clamp(const T& val, const T& min, const T& max) {
  return std::max(min, std::min(val, max));
}
\end{lstlisting}

\end{document}