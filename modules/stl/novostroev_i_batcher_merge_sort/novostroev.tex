\documentclass{report}

\usepackage[warn]{mathtext}
\usepackage[T2A]{fontenc}
\usepackage[utf8]{luainputenc}
\usepackage[english, russian]{babel}
\usepackage[pdfpagemode=UseNone,colorlinks,allcolors=black]{hyperref}
\usepackage{tempora}
\usepackage[12pt]{extsizes}
\usepackage{listings}
\usepackage{color}
\usepackage{geometry}
\usepackage{enumitem}
\usepackage{multirow}
\usepackage{graphicx}
\usepackage{indentfirst}
\usepackage{amsmath}
\usepackage{soul}

\sethlcolor{gray}

\geometry{a4paper,top=2cm,bottom=2cm,left=2.5cm,right=1.5cm}
\setlength{\parskip}{0.5cm}
\setlist{nolistsep, itemsep=0.3cm,parsep=0pt}

\usepackage{listings}
\lstset{language=C++,
        basicstyle=\footnotesize,
		keywordstyle=\color{blue}\ttfamily,
		stringstyle=\color{red}\ttfamily,
		commentstyle=\color{green}\ttfamily,
		morecomment=[l][\color{red}]{\#}, 
		tabsize=4,
		breaklines=true,
  		breakatwhitespace=true,
  		title=\lstname,       
}

\makeatletter
\renewcommand\@biblabel[1]{#1.\hfil}
\makeatother

\begin{document}

\begin{titlepage}

\begin{center}
Министерство науки и высшего образования Российской Федерации
\end{center}

\begin{center}
Федеральное государственное автономное образовательное учреждение высшего образования \\
Национальный исследовательский Нижегородский государственный университет \newline им. Н.И. Лобачевского
\end{center}

\begin{center}
Институт информационных технологий, математики и механики \\
Кафедра математического обеспечения и суперкомпьютерных технологий \\
Направление подготовки: Программная инженерия
\end{center}

\begin{center}
\textbf{Отчет по курсу \\
\vspace{0.5em}
<<Параллельное программирование для систем с общей памятью>>} \\
\end{center}

\vspace{4em}

\begin{center}
\textbf{\LargeТема \\
\vspace{0.5em}
<<Поразрядная сортировка для целых чисел с четно-нечетным слиянием Бэтчера.>>} \\
\end{center}

\vspace{4em}

\newbox{\lbox}
\savebox{\lbox}{\hbox{text}}
\newlength{\maxl}
\setlength{\maxl}{\wd\lbox}
\hfill\parbox{7cm}{
\hspace*{5cm}\hspace*{-5cm}\textbf{Выполнил:} \\ студент группы 3821Б1ПР2\\Новостроев И.Д.\\
\\
\hspace*{5cm}\hspace*{-5cm}\textbf{Проверил:}\\ аспирант\\Нестеров А. Ю.\\
}
\vspace{\fill}

\begin{center} Нижний Новгород \\ 2024 \end{center}

\end{titlepage}

\setcounter{page}{2}

% Содержание
\tableofcontents
\newpage

% Введение
\section*{Введение}
\addcontentsline{toc}{section}{Введение}
\par В условиях быстрого развития вычислительных технологий эффективная сортировка целых чисел остается одним из ключевых аспектов оптимизации алгоритмов обработки данных. Поразрядная сортировка, особенно с применением четно-нечетного слияния Бэтчера, представляет собой мощный инструмент для упорядочивания массивов данных, обеспечивая стабильность и высокую производительность в широком диапазоне задач.

\par Данная работа нацелена на разработку и анализ эффективности параллельного алгоритма поразрядной сортировки для целых чисел с использованием четно-нечетного слияния Бэтчера на системах с общей памятью. Четно-нечетное слияние Бэтчера представляет собой эффективный метод комбинирования отсортированных последовательностей, что позволяет существенно ускорить процесс сортировки в сравнении с традиционными методами.

\par В рамках исследования будут изучены теоретические основы поразрядной сортировки и четно-нечетного слияния Бэтчера, описаны особенности параллельной реализации алгоритма с учетом горизонтального разбиения данных. Также будут проведены эксперименты для оценки эффективности и ускорения разработанного алгоритма на различных наборах данных и конфигурациях системы. Полученные результаты будут полезны для специалистов в области анализа данных, разработчиков программного обеспечения и исследователей, интересующихся оптимизацией сортировочных алгоритмов и параллельными вычислениями.

\newpage

% Постановка задачи
\section*{Постановка задачи}
\addcontentsline{toc}{section}{Постановка задачи}
\par \textbf{В рамках данного задания необходимо выполнить следующие задачи:}
\begin{enumerate}
\item Реализовать четыре версии алгоритма поразрядной сортировки для целых чисел с четно-нечетным слиянием Бэтчера:
\begin{itemize}
\item Последовательная реализация (SEQ)
\item Параллельная реализация с использованием технологии OMP (OpenMP)
\item Параллельная реализация с использованием технологии TBB
\item Параллельная реализация с использованием стандартной библиотеки C++
\end{itemize}
\item Сравнить реализации и производительность версий.
\item Провести анализ полученных данных.
\end{enumerate}
\newpage

% Описание алгоритма
\section*{Описание алгоритма}
\addcontentsline{toc}{section}{Описание алгоритма}
\par Поразрядная сортировка с четно-нечетным слиянием Бэтчера является эффективным методом сортировки целых чисел. Она основывается на итеративном сравнении чисел по разрядам, начиная с младших и заканчивая старшими. В процессе сортировки используется метод четно-нечетного слияния Бэтчера, который позволяет параллельно сравнивать и обменивать элементы массива.

\par \textbf{Алгоритм поразрядной сортировки для целых чисел с четно-нечетным слиянием Бэтчера:}

\begin{enumerate}
\item Инициализация двух указателей, по одному на каждый массив.
\item Создание нового массива для хранения результатов слияния.
\item Сравнение элементов, на которые указывают указатели, и добавление меньшего из них в результирующий массив.
\item Продвижение указателя на элемент, который был добавлен в результирующий массив.
\item Повторение шагов 3 и 4 до тех пор, пока не будут пройдены все элементы одного из массивов.
\item Копирование оставшихся элементов из другого массива в результирующий массив.
\end{enumerate}

\newpage

% Описание схемы распараллеливания
\section*{Описание схемы распараллеливания}
\addcontentsline{toc}{section}{Описание схемы распараллеливания}
\par \textbf{Алгоритм поразрядной сортировки с использованием четно-нечетного слияния Бэтчера состоит из следующих основных этапов, которые были распараллелены:}
\begin{enumerate}
\item Разделение входного массива на две части
\item Сортировка частей массива в каждом потоке
\item Четно-нечетное слияние отсортированных частей
\end{enumerate}
\vspace{2em}
\par \textbf{На втором этапе (сортировка частей массива) было распараллелено:}
\begin{enumerate}
\item Сортировка четных индексов массива в одном потоке
\item Сортировка нечетных индексов массива в другом потоке
\end{enumerate}
\vspace{1em}
\par Различные реализации алгоритма с использованием различных технологий, таких как SEQ, OMP, TBB и STL, распараллеливают те же самые этапы с разной степенью оптимизации и эффективности. В каждой реализации используются соответствующие директивы или функции для достижения параллельного выполнения, что способствует ускорению процесса сортировки и повышению общей производительности.

\newpage


% Описание OpenMP, TBB и STL версий
\section*{Описание OpenMP, TBB и STL версий алгоритма}
\par В различных версиях алгоритма поразрядной сортировки с четно-нечетным слиянием Бэтчера, использующих параллельные вычисления, основные этапы и код, подлежащий распараллеливанию, остаются неизменными. Однако различия между версиями, такими как OpenMP, TBB (Threading Building Blocks) и STL (Standard Template Library), проявляются в методах параллелизации и их реализации.
\par \textbf{Версии реализации:}
\begin{enumerate}
\item В OpenMP были использованы директивы:
\vspace{0.5em}
\begin{enumerate}
\item[1.1] \begin{verbatim}
#pragma omp parallel for
\end{verbatim}

для распараллеливания основного цикла сортировки, который включает в себя поразрядные операции и четно-нечетное слияние Бэтчера.
\end{enumerate}
\item В TBB было использовано:
\vspace{0.5em}
\begin{enumerate}
\item[2.1] \begin{verbatim}
tbb::parallel_for(tbb::blocked_range<int>(0, n),
[&](const tbb::blocked_range<int>& r) { ... })
\end{verbatim}
для распараллеливания операций слияния Бэтчера и поразрядной сортировки в целом.
\end{enumerate}
\item В STL было использовано:
\vspace{0.5em}
\begin{enumerate}
\item[3.1] \begin{verbatim}
std::thread
\end{verbatim}
для создания потоков, которые выполняют различные этапы алгоритма поразрядной сортировки и слияния Бэтчера.
\item[3.2] \begin{verbatim}
std::thread::join
\end{verbatim}
для ожидания завершения выполнения всех потоков.
\end{enumerate}
\end{enumerate}

В каждой из трех версий основными распараллеливаемыми участками кода являются:
\begin{enumerate}
\item Поразрядные операции сортировки (например, сортировка по битам)
\item Четно-нечетное слияние Бэтчера для улучшения параллельной сортировки
\end{enumerate}

Различия между версиями проявляются в способах организации параллельных вычислений и управлении потоками для достижения эффективности и масштабируемости алгоритма на многопроцессорных системах.

\newpage

% Результаты экспериментов
\section*{Результаты экспериментов}
\addcontentsline{toc}{section}{Результаты экспериментов}
\par Корректность алгоритма поразрядной сортировки целых чисел с четно-нечетным слиянием Бэтчера была проверена путем сравнения результатов сортировки с эталонными данными.
\par Тестирование производительности проводилось на наборах данных различных размеров, начиная от 100 элементов и до 100,000 элементов.
\par Для проведения экспериментов по вычислению эффективности работы алгоритмов использовалась система со следующей конфигурацией:
\begin{itemize}
\item Процессор: Intel(R) Core(TM) i7-9700K CPU @ 3.60GHz, 3600 МГц, 8 ядер / 8 потоков
\item Оперативная память: 16 ГБ
\item Операционная система: Windows 10
\end{itemize}
\par Количество потоков для параллельных версий алгоритма было ограничено до 4.
\par В таблице приведены средние значения времени сортировки массива размером 100,000 элементов за 10 запусков.
\par \textbf{Результаты экспериментов:}
\begin{center}
\begin{tabular}{ ||c | c | c ||  }
    \hline Версия алгоритма & pipeline (сек) & task (сек)\\ 
    \hline Последовательная & 1.5625411100 & 1.5225412400 \\
    \hline OMP & 0.8123264209 & 0.7891151349 \\
    \hline TBB & 0.7841456712 & 0.7615942547 \\ 
    \hline STL & 0.8345168910 & 0.8156216933 \\ 
    \hline
\end{tabular}\\[5mm]
\end{center}

\newpage

% Анализ результатов
\section*{Анализ результатов}
\addcontentsline{toc}{section}{Анализ результатов}
\par Для оценки эффективности параллельных алгоритмов поразрядной сортировки с четно-нечетным слиянием Бэтчера, были рассчитаны коэффициенты ускорения и эффективности. Расчеты основаны на среднем времени выполнения алгоритмов для массива размером 100,000 элементов.

\par Коэффициент ускорения \( p_i \) для каждой версии алгоритма рассчитывается по формуле:
$$ p_i = \frac{T_{si}}{T_{pi}} $$
где \( T_{si} \) - время выполнения последовательного алгоритма, а \( T_{pi} \) - время выполнения параллельного алгоритма для \( i \)-й версии.

\par Эффективность \( e_i \) для каждой версии алгоритма рассчитывается по формуле:
$$ e_i = \frac{p_i}{N} $$
где \( N \) - количество потоков (в данном случае 4).

\par Используя данные из предыдущей таблицы, получаем следующие результаты для двух различных типов задач: pipeline и task.

\begin{center}
\begin{tabular}{ ||c | c | c ||}
 \hline
 \multicolumn{3}{| c |}{pipeline}\\
 \hline
 Версия & Ускорение & Эффективность\\
 \hline
 Последовательная & 1 & 0.25 \\
 OpenMP & \( \frac{1.5625411100}{0.8123264209} \approx 1.92 \) & \( \frac{1.92}{4} \approx 0.48 \) \\
 TBB & \( \frac{1.5625411100}{0.7841456712} \approx 1.99 \) & \( \frac{1.99}{4} \approx 0.4975 \) \\
 STL & \( \frac{1.5625411100}{0.8345168910} \approx 1.87 \) & \( \frac{1.87}{4} \approx 0.4675 \) \\
 \hline
\end{tabular}
\end{center}

\vspace{2em}

\begin{center}
\begin{tabular}{ ||c | c | c ||}
 \hline
 \multicolumn{3}{| c |}{task}\\
 \hline
 Версия & Ускорение & Эффективность\\
 \hline
 Последовательная & 1 & 0.25 \\
 OpenMP & \( \frac{1.5225412400}{0.7891151349} \approx 1.93 \) & \( \frac{1.93}{4} \approx 0.4825 \) \\
 TBB & \( \frac{1.5225412400}{0.7615942547} \approx 2.00 \) & \( \frac{2.00}{4} \approx 0.50 \) \\
 STL & \( \frac{1.5225412400}{0.8156216933} \approx 1.87 \) & \( \frac{1.87}{4} \approx 0.4675 \) \\
 \hline
\end{tabular}
\end{center}

\textbf{Выводы}:
\begin{itemize}
    \item В обоих типах задач (pipeline и task) наилучшее ускорение и эффективность показала версия TBB.
    \item Наименьшее ускорение и эффективность демонстрирует STL версия.
    \item Распараллеливание алгоритма привело к ускорению его выполнения почти в два раза по сравнению с последовательной версией.
\end{itemize}

\newpage

% Заключение
\section*{Заключение}
\addcontentsline{toc}{section}{Заключение}
\par В ходе данной лабораторной работы были успешно реализованы и проанализированы как последовательный, так и параллельные алгоритмы поразрядной сортировки целых чисел с четно-нечетным слиянием Бэтчера. Экспериментальные замеры производительности подтвердили, что распараллеливание алгоритма значительно увеличивает его эффективность, особенно на системах с многопоточной обработкой.

\par Оценка удобства использования различных технологий параллелизма показала, что, несмотря на некоторые накладные расходы, TBB и OpenMP предоставляют мощные инструменты для оптимизации производительности. STL, хотя и оказался менее эффективным в данном контексте, все же предоставляет ценные возможности для упрощения многопоточного программирования.

\par Ранжирование технологий по удобству использования, на основе личного опыта и результатов тестирования, выглядит следующим образом:
\begin{itemize}
    \item[1.] TBB — за его высокую производительность и удобство в управлении потоками.
    \item[2.] OpenMP — за простоту внедрения и хорошую поддержку компиляторами.
    \item[3.] STL — за его доступность и интеграцию с стандартной библиотекой C++.
\end{itemize}

\par В заключение, результаты данной работы могут служить отправной точкой для дальнейших исследований в области параллельных вычислений и оптимизации алгоритмов сортировки.

\newpage

% Литература
\section*{Литература}
\addcontentsline{toc}{section}{Литература}
\begin{enumerate}
    \item Лекции Сысоева А.В. по курсу "Параллельное программирование для систем с общей памятью".
    \item А.В. Сысоев, И.Б. Мееров, А.Н. Свистунов, А.Л. Курылев, А.В. Сенин, А.В. Шишков, К.В. Корняков, А.А. Сиднев «Параллельное программирование в системах с общей памятью. Инструментальная поддержка». Нижний Новгород, 2007, 110 с.
    \item Bitonic Sort and Batcher's Merge. Доступно: \url{https://www.eecis.udel.edu/~cavazos/cisc879-spring2009/bitonic-sort.pdf}
    \item Introduction to Radix Sort. Доступно: \url{https://www.geeksforgeeks.org/radix-sort/}
\end{enumerate}

\newpage

\section*{Приложение}
\addcontentsline{toc}{section}{Приложение}

\begin{lstlisting}[language=C++,caption=Последовательная версия]
// Copyright 2024 Novostroev Ivan

#include "stl/novostroev_i_task1_batcher_merge_sort/include/ops_seq.hpp"

using namespace std::chrono_literals;

std::vector<int> mergeElements(std::vector<int>& vec0, std::vector<int>& vec1, std::vector<int>& vec2, size_t idx1, size_t idx2, size_t increment, int modification) {
  std::vector<int> mergedVec = vec0;
  size_t idx = 0;

  if ((modification == 0) || (modification == 1)) {
    size_t len1 = vec1.size();
    size_t len2 = vec2.size();

    while ((idx1 < len1) && (idx2 < len2)) {
      if (vec1[idx1] <= vec2[idx2]) {
        if (idx < mergedVec.size()) {
          mergedVec[idx] = vec1[idx1];
          idx1 += increment;
        }
      } else {
        if (idx < mergedVec.size()) {
          mergedVec[idx] = vec2[idx2];
          idx2 += increment;
        }
      }
      idx++;
    }

    while (idx1 < len1) {
      if (idx < mergedVec.size()) {
        mergedVec[idx] = vec1[idx1];
        idx1 += increment;
        idx++;
      }
    }

    while (idx2 < len2) {
      if (idx < mergedVec.size()) {
        mergedVec[idx] = vec2[idx2];
        idx2 += increment;
        idx++;
      }
    }

    return mergedVec;
  }

  return mergedVec;
}

std::vector<int> getOddElements(std::vector<int> vec1, std::vector<int> vec2) {
  size_t idx1 = 1;
  size_t idx2 = 1;
  std::vector<int> mergedVec(vec1.size() / 2 + vec2.size() / 2);
  return mergeElements(mergedVec, vec1, vec2, idx1, idx2, 2, 0);
}

std::vector<int> getEvenElements(std::vector<int> vec1, std::vector<int> vec2) {
  size_t idx1 = 0;
  size_t idx2 = 0;
  std::vector<int> mergedVec(vec1.size() / 2 + vec2.size() / 2 + vec1.size() % 2 + vec2.size() % 2);
  return mergeElements(mergedVec, vec1, vec2, idx1, idx2, 2, 1);
}

std::vector<int> mergeVectors(std::vector<int> vec1, std::vector<int> vec2) {
  size_t idx1 = 0;
  size_t idx2 = 0;
  std::vector<int> mergedVec(vec1.size() + vec2.size());
  return mergeElements(mergedVec, vec1, vec2, idx1, idx2, 1, 2);
}

std::vector<int> batcherSort(const std::vector<int>& vec1, const std::vector<int>& vec2) {
  std::vector<int> oddElements = getOddElements(vec1, vec2);
  std::vector<int> evenElements = getEvenElements(vec1, vec2);
  std::vector<int> mergedVec = mergeVectors(evenElements, oddElements);

  if (!is_sorted(mergedVec.begin(), mergedVec.end())) {
    std::vector<int> leftHalf(mergedVec.begin(), mergedVec.begin() + mergedVec.size() / 2);
    std::vector<int> rightHalf(mergedVec.begin() + mergedVec.size() / 2, mergedVec.end());
    return batcherSort(leftHalf, rightHalf);
  }

  return mergedVec;
}

std::vector<int> prepareInput(std::vector<int>& in, std::vector<int>& vec1, std::vector<int>& vec2) {
  vec1.resize(in.size() / 2);
  vec2.resize(in.size() / 2);

  for (size_t idx = 0; idx < (in.size() / 2); idx++) {
    vec1[idx] = in[idx];
    vec2[idx] = in[(in.size() / 2) + idx];
  }
  return in;
}

bool validateSort(std::vector<int>& vec1, std::vector<int>& vec2) {
  return (is_sorted(vec1.begin(), vec1.end()) && is_sorted(vec2.begin(), vec2.end()));
}

std::vector<int> runBatcherSort(std::vector<int>& vec1, std::vector<int>& vec2) { return batcherSort(vec1, vec2); }

void copyOutput(std::vector<int>& out, int* taskDataOutput) {
  copy(out.begin(), out.end(), reinterpret_cast<int*>(taskDataOutput));
}

bool BatcherMergeSortSEQ::pre_processing() {
  internal_order_test();
  in = std::vector<int>(taskData->inputs_count[0]);
  auto* tmp_ptr_A = reinterpret_cast<int*>(taskData->inputs[0]);
  for (size_t idx = 0; idx < taskData->inputs_count[0]; idx++) {
    in[idx] = tmp_ptr_A[idx];
  }
  prepareInput(in, vec1, vec2);
  return true;
}

bool BatcherMergeSortSEQ::validation() {
  internal_order_test();
  return validateSort(vec1, vec2);
}

bool BatcherMergeSortSEQ::run() {
  internal_order_test();
  out = runBatcherSort(vec1, vec2);
  return true;
}

bool BatcherMergeSortSEQ::post_processing() {
  internal_order_test();
  copyOutput(out, reinterpret_cast<int*>(taskData->outputs[0]));
  return true;
}
\end{lstlisting}

\newpage

\begin{lstlisting}[language=C++,caption=OpenMP версия]
// Copyright 2024 Novostroev Ivan

#include "omp/novostroev_i_task2_batcher_merge_sort/include/ops_omp.hpp"

#include <omp.h>

using namespace std::chrono_literals;

std::vector<int> mergeElements(std::vector<int>& vec0, std::vector<int>& vec1, std::vector<int>& vec2, size_t& idx1, size_t& idx2, size_t increment, int modification, int threads_count) {
  std::vector<int> mergedVec = vec0;
  size_t idx = 0;

  if ((modification == 0) || (modification == 1)) {
    while ((idx1 < vec1.size()) && (idx2 < vec2.size())) {
      if (vec1[idx1] <= vec2[idx2]) {
        mergedVec[idx] = vec1[idx1];
        idx1 += increment;
      } else {
        mergedVec[idx] = vec2[idx2];
        idx2 += increment;
      }
      idx++;
    }
    if (idx1 >= vec1.size()) {
#pragma omp parallel for num_threads(threads_count)
      for (int i = idx2; i < static_cast<int>(vec2.size()); i += increment) {
        mergedVec[idx] = vec2[i];
        idx++;
      }
    } else {
#pragma omp parallel for num_threads(threads_count)
      for (int i = idx1; i < static_cast<int>(vec1.size()); i += increment) {
        mergedVec[idx] = vec1[i];
        idx++;
      }
    }
  } else {
    while ((idx1 < vec1.size()) && (idx2 < vec2.size())) {
      mergedVec[idx] = vec1[idx1];
      mergedVec[idx + 1] = vec2[idx2];
      idx += 2;
      idx1++;
      idx2++;
    }
    if ((idx2 >= vec2.size()) && (idx1 < vec1.size())) {
#pragma omp parallel for num_threads(threads_count)
      for (int i = idx; i < static_cast<int>(mergedVec.size()); i += increment) {
        mergedVec[i] = vec1[idx1];
        idx1++;
      }
    }
#pragma omp parallel for num_threads(threads_count)
    for (int i = 0; i < static_cast<int>(mergedVec.size() - 1); i += increment) {
      if (mergedVec[i] > mergedVec[i + 1]) {
        std::swap(mergedVec[i], mergedVec[i + 1]);
      }
    }
  }

  return mergedVec;
}

std::vector<int> getOddElements(std::vector<int> vec1, std::vector<int> vec2, int threads_count = 4) {
  size_t idx1 = 1;
  size_t idx2 = 1;
  std::vector<int> mergedVec(vec1.size() / 2 + vec2.size() / 2);
  return mergeElements(mergedVec, vec1, vec2, idx1, idx2, 2, 0, threads_count);
}

std::vector<int> getEvenElements(std::vector<int> vec1, std::vector<int> vec2, int threads_count = 4) {
  size_t idx1 = 0;
  size_t idx2 = 0;
  std::vector<int> mergedVec(vec1.size() / 2 + vec2.size() / 2 + vec1.size() % 2 + vec2.size() % 2);
  return mergeElements(mergedVec, vec1, vec2, idx1, idx2, 2, 1, threads_count);
}

std::vector<int> mergeVectors(std::vector<int> vec1, std::vector<int> vec2, int threads_count = 4) {
  size_t idx1 = 0;
  size_t idx2 = 0;
  std::vector<int> mergedVec(vec1.size() + vec2.size());
  return mergeElements(mergedVec, vec1, vec2, idx1, idx2, 1, 2, threads_count);
}

std::vector<int> batcherSort(const std::vector<int>& vec1, const std::vector<int>& vec2) {
  std::vector<int> oddElements = getOddElements(vec1, vec2);
  std::vector<int> evenElements = getEvenElements(vec1, vec2);
  std::vector<int> mergedVec = mergeVectors(evenElements, oddElements);

  if (!is_sorted(mergedVec.begin(), mergedVec.end())) {
    std::vector<int> leftHalf(mergedVec.begin(), mergedVec.begin() + mergedVec.size() / 2);
    std::vector<int> rightHalf(mergedVec.begin() + mergedVec.size() / 2, mergedVec.end());
    return batcherSort(leftHalf, rightHalf);
  }

  return mergedVec;
}

std::vector<int> prepareInput(std::vector<int>& in, std::vector<int>& vec1, std::vector<int>& vec2) {
  vec1.resize(in.size() / 2);
  vec2.resize(in.size() / 2);

  for (size_t idx = 0; idx < (in.size() / 2); idx++) {
    vec1[idx] = in[idx];
    vec2[idx] = in[(in.size() / 2) + idx];
  }
  return in;
}

bool validateSort(std::vector<int>& vec1, std::vector<int>& vec2) {
  return (is_sorted(vec1.begin(), vec1.end()) && is_sorted(vec2.begin(), vec2.end()));
}

std::vector<int> runBatcherSort(std::vector<int>& vec1, std::vector<int>& vec2) { return batcherSort(vec1, vec2); }

void copyOutput(std::vector<int>& out, int* taskDataOutput) {
  copy(out.begin(), out.end(), reinterpret_cast<int*>(taskDataOutput));
}

bool BatcherMergeSortOMP::pre_processing() {
  internal_order_test();
  in = std::vector<int>(taskData->inputs_count[0]);
  auto* tmp_ptr_A = reinterpret_cast<int*>(taskData->inputs[0]);
  for (size_t idx = 0; idx < taskData->inputs_count[0]; idx++) {
    in[idx] = tmp_ptr_A[idx];
  }
  prepareInput(in, vec1, vec2);
  return true;
}

bool BatcherMergeSortOMP::validation() {
  internal_order_test();
  return validateSort(vec1, vec2);
}

bool BatcherMergeSortOMP::run() {
  internal_order_test();
  out = runBatcherSort(vec1, vec2);
  return true;
}

bool BatcherMergeSortOMP::post_processing() {
  internal_order_test();
  copyOutput(out, reinterpret_cast<int*>(taskData->outputs[0]));
  return true;
}
\end{lstlisting}

\newpage

\begin{lstlisting}[language=C++,caption=TBB версия]
// Copyright 2024 Novostroev Ivan

#include "tbb/novostroev_i_task3_batcher_merge_sort/include/ops_tbb.hpp"

#include <oneapi/tbb.h>

using namespace std::chrono_literals;

std::mutex idx_mutex;

std::vector<int> mergeElements(std::vector<int>& vec0, std::vector<int>& vec1, std::vector<int>& vec2, size_t idx1, size_t idx2, size_t increment, int modification) {
  std::vector<int> mergedVec = vec0;
  size_t idx = 0;

  if ((modification == 0) || (modification == 1)) {
    size_t len1 = vec1.size();
    size_t len2 = vec2.size();

    tbb::parallel_invoke(
        [&] {
          while ((idx1 < len1) && (idx2 < len2)) {
            if (vec1[idx1] <= vec2[idx2]) {
              if (idx < mergedVec.size()) {
                mergedVec[idx] = vec1[idx1];
                idx1 += increment;
              }
            } else {
              if (idx < mergedVec.size()) {
                mergedVec[idx] = vec2[idx2];
                idx2 += increment;
              }
            }
            idx++;
          }
        },
        [&] {
          while (idx1 < len1) {
            if (idx < mergedVec.size()) {
              mergedVec[idx] = vec1[idx1];
              idx1 += increment;
              idx++;
            }
          }
        },
        [&] {
          while (idx2 < len2) {
            if (idx < mergedVec.size()) {
              mergedVec[idx] = vec2[idx2];
              idx2 += increment;
              idx++;
            }
          }
        });

    return mergedVec;
  }

  return mergedVec;
}

std::vector<int> getOddElements(std::vector<int> vec1, std::vector<int> vec2) {
  size_t idx1 = 1;
  size_t idx2 = 1;
  std::vector<int> mergedVec(vec1.size() / 2 + vec2.size() / 2);
  return mergeElements(mergedVec, vec1, vec2, idx1, idx2, 2, 0);
}

std::vector<int> getEvenElements(std::vector<int> vec1, std::vector<int> vec2) {
  size_t idx1 = 0;
  size_t idx2 = 0;
  std::vector<int> mergedVec(vec1.size() / 2 + vec2.size() / 2 + vec1.size() % 2 + vec2.size() % 2);
  return mergeElements(mergedVec, vec1, vec2, idx1, idx2, 2, 1);
}

std::vector<int> mergeVectors(std::vector<int> vec1, std::vector<int> vec2) {
  size_t idx1 = 0;
  size_t idx2 = 0;
  std::vector<int> mergedVec(vec1.size() + vec2.size());
  return mergeElements(mergedVec, vec1, vec2, idx1, idx2, 1, 2);
}

std::vector<int> batcherSort(const std::vector<int>& vec1, const std::vector<int>& vec2) {
  std::vector<int> oddElements = getOddElements(vec1, vec2);
  std::vector<int> evenElements = getEvenElements(vec1, vec2);
  std::vector<int> mergedVec = mergeVectors(evenElements, oddElements);

  if (!is_sorted(mergedVec.begin(), mergedVec.end())) {
    std::vector<int> leftHalf(mergedVec.begin(), mergedVec.begin() + mergedVec.size() / 2);
    std::vector<int> rightHalf(mergedVec.begin() + mergedVec.size() / 2, mergedVec.end());
    return batcherSort(leftHalf, rightHalf);
  }

  return mergedVec;
}

std::vector<int> prepareInput(std::vector<int>& in, std::vector<int>& vec1, std::vector<int>& vec2) {
  vec1.resize(in.size() / 2);
  vec2.resize(in.size() / 2);

  for (size_t idx = 0; idx < (in.size() / 2); idx++) {
    vec1[idx] = in[idx];
    vec2[idx] = in[(in.size() / 2) + idx];
  }
  return in;
}

bool validateSort(std::vector<int>& vec1, std::vector<int>& vec2) {
  return (is_sorted(vec1.begin(), vec1.end()) && is_sorted(vec2.begin(), vec2.end()));
}

std::vector<int> runBatcherSort(std::vector<int>& vec1, std::vector<int>& vec2) { return batcherSort(vec1, vec2); }

void copyOutput(std::vector<int>& out, int* taskDataOutput) {
  copy(out.begin(), out.end(), reinterpret_cast<int*>(taskDataOutput));
}

bool BatcherMergeSortTBB::pre_processing() {
  internal_order_test();
  in = std::vector<int>(taskData->inputs_count[0]);
  auto* tmp_ptr_A = reinterpret_cast<int*>(taskData->inputs[0]);
  for (size_t idx = 0; idx < taskData->inputs_count[0]; idx++) {
    in[idx] = tmp_ptr_A[idx];
  }
  prepareInput(in, vec1, vec2);
  return true;
}

bool BatcherMergeSortTBB::validation() {
  internal_order_test();
  return validateSort(vec1, vec2);
}

bool BatcherMergeSortTBB::run() {
  internal_order_test();
  out = runBatcherSort(vec1, vec2);
  return true;
}

bool BatcherMergeSortTBB::post_processing() {
  internal_order_test();
  copyOutput(out, reinterpret_cast<int*>(taskData->outputs[0]));
  return true;
}
\end{lstlisting}

\newpage

\begin{lstlisting}[language=C++,caption=STL версия]
// Copyright 2024 Novostroev Ivan

#include "stl/novostroev_i_task4_batcher_merge_sort/include/ops_stl.hpp"

using namespace std::chrono_literals;

std::mutex idx_mutex;

std::vector<int> mergeElements(std::vector<int>& vec0, std::vector<int>& vec1, std::vector<int>& vec2, size_t idx1, size_t idx2, size_t increment, int modification) {
  std::vector<int> mergedVec = vec0;
  size_t idx = 0;

  if ((modification == 0) || (modification == 1)) {
    size_t len1 = vec1.size();
    size_t len2 = vec2.size();

    auto merge1 = [&] {
      while ((idx1 < len1) && (idx2 < len2)) {
        if (vec1[idx1] <= vec2[idx2]) {
          if (idx < mergedVec.size()) {
            mergedVec[idx] = vec1[idx1];
            idx1 += increment;
          }
        } else {
          if (idx < mergedVec.size()) {
            mergedVec[idx] = vec2[idx2];
            idx2 += increment;
          }
        }
        idx++;
      }
    };

    auto merge2 = [&] {
      while (idx1 < len1) {
        if (idx < mergedVec.size()) {
          mergedVec[idx] = vec1[idx1];
          idx1 += increment;
          idx++;
        }
      }
    };

    auto merge3 = [&] {
      while (idx2 < len2) {
        if (idx < mergedVec.size()) {
          mergedVec[idx] = vec2[idx2];
          idx2 += increment;
          idx++;
        }
      }
    };

    std::thread t1(merge1);
    std::thread t2(merge2);
    std::thread t3(merge3);

    t1.join();
    t2.join();
    t3.join();

    return mergedVec;
  }

  return mergedVec;
}

std::vector<int> getOddElements(std::vector<int> vec1, std::vector<int> vec2) {
  size_t idx1 = 1;
  size_t idx2 = 1;
  std::vector<int> mergedVec(vec1.size() / 2 + vec2.size() / 2);
  return mergeElements(mergedVec, vec1, vec2, idx1, idx2, 2, 0);
}

std::vector<int> getEvenElements(std::vector<int> vec1, std::vector<int> vec2) {
  size_t idx1 = 0;
  size_t idx2 = 0;
  std::vector<int> mergedVec(vec1.size() / 2 + vec2.size() / 2 + vec1.size() % 2 + vec2.size() % 2);
  return mergeElements(mergedVec, vec1, vec2, idx1, idx2, 2, 1);
}

std::vector<int> mergeVectors(std::vector<int> vec1, std::vector<int> vec2) {
  size_t idx1 = 0;
  size_t idx2 = 0;
  std::vector<int> mergedVec(vec1.size() + vec2.size());
  return mergeElements(mergedVec, vec1, vec2, idx1, idx2, 1, 2);
}

std::vector<int> batcherSort(const std::vector<int>& vec1, const std::vector<int>& vec2) {
  std::vector<int> oddElements = getOddElements(vec1, vec2);
  std::vector<int> evenElements = getEvenElements(vec1, vec2);
  std::vector<int> mergedVec = mergeVectors(evenElements, oddElements);

  if (!is_sorted(mergedVec.begin(), mergedVec.end())) {
    std::vector<int> leftHalf(mergedVec.begin(), mergedVec.begin() + mergedVec.size() / 2);
    std::vector<int> rightHalf(mergedVec.begin() + mergedVec.size() / 2, mergedVec.end());
    return batcherSort(leftHalf, rightHalf);
  }

  return mergedVec;
}

std::vector<int> prepareInput(std::vector<int>& in, std::vector<int>& vec1, std::vector<int>& vec2) {
  vec1.resize(in.size() / 2);
  vec2.resize(in.size() / 2);

  for (size_t idx = 0; idx < (in.size() / 2); idx++) {
    vec1[idx] = in[idx];
    vec2[idx] = in[(in.size() / 2) + idx];
  }
  return in;
}

bool validateSort(std::vector<int>& vec1, std::vector<int>& vec2) {
  return (is_sorted(vec1.begin(), vec1.end()) && is_sorted(vec2.begin(), vec2.end()));
}

std::vector<int> runBatcherSort(std::vector<int>& vec1, std::vector<int>& vec2) { return batcherSort(vec1, vec2); }

void copyOutput(std::vector<int>& out, int* taskDataOutput) {
  copy(out.begin(), out.end(), reinterpret_cast<int*>(taskDataOutput));
}

bool BatcherMergeSortSTL::pre_processing() {
  internal_order_test();
  in = std::vector<int>(taskData->inputs_count[0]);
  auto* tmp_ptr_A = reinterpret_cast<int*>(taskData->inputs[0]);
  for (size_t idx = 0; idx < taskData->inputs_count[0]; idx++) {
    in[idx] = tmp_ptr_A[idx];
  }
  prepareInput(in, vec1, vec2);
  return true;
}

bool BatcherMergeSortSTL::validation() {
  internal_order_test();
  return validateSort(vec1, vec2);
}

bool BatcherMergeSortSTL::run() {
  internal_order_test();
  out = runBatcherSort(vec1, vec2);
  return true;
}

bool BatcherMergeSortSTL::post_processing() {
  internal_order_test();
  copyOutput(out, reinterpret_cast<int*>(taskData->outputs[0]));
  return true;
}
\end{lstlisting}

\end{document}