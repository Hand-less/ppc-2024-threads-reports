\documentclass[a4paper]{article}
\usepackage[T2A]{fontenc}
\usepackage[utf8]{inputenc}
\usepackage[english,russian]{babel}
\usepackage{listings}
\usepackage[colorlinks=false]{hyperref}
\usepackage{indentfirst}
\usepackage{multirow}
\usepackage{enumitem}
\usepackage{titlesec}
\usepackage{secdot}
\usepackage{graphicx}
\usepackage{titletoc}
\graphicspath{ {./images/} }
\usepackage{tocloft}
\titlelabel{\thetitle. \quad}
\usepackage{geometry}
\geometry{left=2.5cm}
\geometry{right=1.5cm}
\geometry{top=2cm}
\geometry{bottom=2cm}

\titlespacing*{\section}{0em}{0ex}{1em}

\usepackage{color}
\definecolor{dkgreen}{rgb}{0,0.6,0}
\definecolor{gray}{rgb}{0.5,0.5,0.5}
\definecolor{mauve}{rgb}{0.58,0, 0.82}

\lstset{
language=C++,
aboveskip=3mm,
xleftmargin=10mm,
belowskip=3mm,
showstringspaces=false,
columns=flexible,
basicstyle={\ttfamily},
numbers=none,
numberstyle=\small\color{gray},
keywordstyle=\color{blue},
commentstyle=\color{dkgreen},
stringstyle=\color{mauve},
breaklines=true,
breakatwhitespace=true,
tabsize=3
}

\renewcommand{\thesection}{\arabic{section}}
\renewcommand{\cftsecleader}{\cftdotfill{\cftdotsep}}

\begin{document}
\begin{titlepage}
 \begin{center}
{\small МИНИСТЕРСТВО НАУКИ И ВЫСШЕГО ОБРАЗОВАНИЯ РОССИЙСКОЙ ФЕДЕРАЦИИ\\
 Федеральное государственное автономное образовательное учреждение
высшего образования
}
\textbf{<<Национальный исследовательский
Нижегородский государственный университет им. Н.И. Лобачевского>>\\
(ННГУ)\\}

\vspace*{1cm}

\textbf{Институт информационных технологий, математики и механики}

\vspace*{3cm}

\textbf{\LargeОТЧЁТ}

по лабораторной работе

по курсу <<Параллельное программирование для систем с общей памятью>>

\vspace{1em}

\textbf{\large<<Построение выпуклой оболочки -- проход Грэхема>>}

\end{center}

\vspace{2cm}

\begin{flushright}
\parbox{8.5cm}{
\textbf{Выполнил:}

студент группы 3821Б1ПР1\\
Крисеев М.А.
}
\end{flushright}

\flushbottom

\vfill

\begin{center}
 Нижний Новгород

 \the\year
\end{center}

\end{titlepage}

\tableofcontents

\newpage
\section*{Введение}
\label{sec:intro}
\addcontentsline{toc}{section}{Введение}

В вычислительной геометрии выпуклые фигуры занимают очень важное место. 
Для нахождения таких фигур из множества точек существует достаточно много алгоритмов
т.н. <<построения минимальной выпуклой оболочки>>. Такая оболочка является минимально возможной
выпуклой фигурой (замкнутой кривой), охватывающей все точки множества. Фактически,
построение такой оболочки сводится к выбору и упорядочению из всего множества точек тех, которые 
входят в искомую оболочку.
\newpage
\section{Ход работы}
\subsection{Постановка задачи}

Моей задачей является реализация алгоритма построения выпуклой оболочки 
проходом Грэхема и распараллеливание данного алгоритма при помощи различных технологий:

\begin{itemize}
    \item OpenMP
    \item Intel Thread Building Blocks (TBB)
    \item C++ Standard library
\end{itemize}

Также мне необходимо сравнить реализации и производительность этих
версий и проанализировать полученные данные.

\subsection{Описание алгоритма}

На первом шаге алгоритма необходимо найти начальную точку, гарантированно входящую
в искомую оболочку. Такой точкой является точка с минимальной ординатой.

Далее все остальные точки сортируются по их полярному углу в системе с центром
в начальной точке. 

После этого необходимо занести начальную точку и первые две точки из отсортированного массива занести в стек, 
после чего для каждой следующей точки проверяется угол, проходящий через неё, вершину стека и вторую точку сверху стека, и если 
образуется правый поворот -- вершина удаляется. Как только образуется левый поворот или останется только одна точка в стеке --
выбранная точка помещается в стек. 

\subsection{Схема распараллеливания}

Поскольку асимптотически наиболее вычислительно сложной операцией является сортировка массива точек,
а объединение оболочек достаточно сложно, что компенсирует ускорение засчёт параллельного вычисления 
меньшей оболочки одним потоком и не даёт значительного прироста по сравнению с линейным проходом -- 
наиболее удобной точкой распараллеливания будет именно сортировка.

Последовательная сортировка std::sort работает очень быстро (позволяя обрабатывать десятки миллионов точек за $\approx 15$ с), 
а последовательный проход после неё работает ещё в несколько раз быстрее. 
Распараллелить стандартную сортировку я не могу, поэтому я вынужден писать свой собственный алгоритм сортировки. 
Я использовал odd-even сортировку.

\subsection{Описание OpenMP версии}

Поскольку сортировка odd-even создана для параллельных вычислений, 
её параллельная версия на OMP реализуется тривиально добавлением 
трёх директив.

Перед внешним циклом:
\begin{lstlisting}
#pragma omp parallel
\end{lstlisting}

Перед обоими вложенными циклами:

\begin{lstlisting}
#pragma omp for
\end{lstlisting}

\subsection{Описание TBB версии}

Для TBB реализация несколько сложнее, поскольку, в отличие от OMP эта библиотека
требует реструктуризации кода. 

Для распараллеливания с TBB я заменил внутренние циклы на tbb::parallel\_for.
В качестве итерационного пространства я использовал стандартный tbb::blocked\_range<size\_t> с 
grain\_size равным $1/16$ размера массива.

\subsection{Описание версии с std::thread}
Для использования std::thread понадобилось еще сильнее реструктурировать код. 
Я вынес код сортировки в отдельную функцию, которая разделяет всё множество точек во внутренних циклах поровну на все потоки.

Чтобы не нарушать четность при распараллеливании, пришлось добавить дополнительные условия в начале 
функции для каждого потока. 

Внешний цикл также имеется в каждом потоке. При завершении итерации внешнего цикла происходит синхронизация
при помощи std::barrier.

\subsection{Результаты экспериментов}

Эксперименты я проводил на компьютере с ЦПУ AMD Ryzen 7 5700X (16 доступных потоков). Запуск производился на 
ОС Arch Linux, ядро Linux Zen 6.9.1, компилятор G++ версии 14.1.1 20240522.
При тестах на производительность использовались входные данные в виде 5000 точек, сгенерированных 
генератором псевдослучайных чисел std::mt19937 с сидом 1587443 с равномерным распределением
на отрезке $[-2.5; 4]$. Сборка CMAKE в конфигурации Release с оптимизацией -O3.

Результаты экспериментов следующие:


\begin{center}
    \begin{tabular}{ |c|c|c|c|c| }
        \hline
        Тип запуска & Последовательная & OpenMP & TBB & STL \\
        \hline
        Task only & 315.408 мс & 61.330 мс & 218.310 мс & 256.494 мс \\
        \hline
        Pipeline & 358.709 мс & 122.241 мс & 235.657 мс & 276.540 \\
        \hline
    \end{tabular}
\end{center}

Корректность эксперимента подтверждается тем, что входные данные у всех методов одинаковы, 
программы запускаются на одном и том же устройстве с одной и той же конфигурацией оборудования и ОС и одним
и тем же компилятором. 

Чтобы снизить влияние отклонений в связи с неидеальностью условий проведения -- данные приведены 
в виде среднего арифметического результатов 10 запусков. 

После каждого запуска также проверяется корректность полученного результата выполнения.

Вычислим полученное ускорение и эффективность по следующим формулам.

Ускорение:

$$p = \frac{T_{seq}}{T_{par}}$$
где $T_{seq}$ и $T_{par}$ -- времена выполнения последовательной и параллельной версии соответственно.

Эффективность:

$$e = \frac{p}{n}$$
где $p$ -- ускорение, $n$ -- число потоков, в моём случае $n = 16$.

Получаем следующие результаты:

\smallskip
Task only:

\begin{center}
    \begin{tabular}{ |c|c|c| }
        \hline
        Версия & Ускорение & Эффективность \\
        \hline
        Последовательная & 1 & 0.0625 \\
        OpenMP & 5.143 & 0.321 \\
        TBB & 1.445 & 0.090 \\
        STL & 1.230 & 0.076 \\
        \hline
    \end{tabular}
\end{center}

Pipeline:

\begin{center}
    \begin{tabular}{ |c|c|c| }
        \hline
        Версия & Ускорение & Эффективность \\
        \hline
        Последовательная & 1 & 0.0625 \\
        OpenMP & 2.934 & 0.183 \\
        TBB & 1.522 & 0.095 \\
        STL & 1.297 & 0.081 \\
        \hline
    \end{tabular}
\end{center}

\subsection{Выводы}

Распараллеливание данного алгоритма хоть и незначительно, но ускорило его работу. 
Параллельная версия с STL потоками с этим справилась хуже всего, с OpenMP -- лучше всего, ускорив работу более, чем в 5 раз. 
Остальные технологии ускорили работу лишь в 1.5 раза. 
\newpage
\section*{Заключение}
\addcontentsline{toc}{section}{Заключение}

В рамках данной лабораторной работы я изучил алгоритм построения выпуклой оболочки 
проходом Грэхема и исследовал возможности его распараллеливания. Научился работать с технологией OpenMP, библиотекой Intel Thread Building Blocks (TBB) и потоками стандартной библиотеки языка C++.
Реализовал при помощи этих технологий параллельные версии данного алгоритма и исследовал их.

Наиболее эффективной и удобной в использовании для меня оказалась OpenMP.
\newpage
\section*{Использованная литература}

\begin{enumerate}
    \item Сысоев А.В. Лекции по курсу "Параллельное программирование для систем с общей памятью"
    \item Построение минимальных выпуклых оболочек // Habr -- Режим доступа: \url{https://habr.com/ru/articles/144921/}
    \item Intel® oneAPI Threading Building Blocks Cookbook -- Режим доступа: \url{https://www.intel.com/content/www/us/en/docs/onetbb/cookbook/2021-6/overview.html}
\end{enumerate}
\appendix
\renewcommand{\contentslabel}{Приложение }
\titleformat{\section}{\newpage\normalfont\Large\bfseries}{\contentslabel\thesection.}{1em}{}

\titlecontents{section}
              [0em]
              {}%
              {\textbf{Приложение \thecontentslabel}.\space}%
              {\bfseries}%
              {\cftdotfill{\cftdotsep} \bfseries \contentspage}%

\section{Код последовательной версии}


\begin{lstlisting}
#include "include/kriseev_m_convex_hull_seq.hpp"

#include <algorithm>
#include <limits>

double angle1(const KriseevMTaskSeq::Point &origin, const KriseevMTaskSeq::Point &point) {
  double dx = point.first - origin.first;
  double dy = point.second - origin.second;
  if (dx == 0.0 && dy == 0.0) {
    // Return this incorrect value to ensure
    // the origin is always first in sorted array
    return -1.0;
  }
  // Calculating angle using algorithm from
  // https://stackoverflow.com/a/14675998/14116050
  return (dy >= 0) ? (dx >= 0 ? dy / (dx + dy) : 1 - dx / (-dx + dy))
                   : (dx < 0 ? 2 - dy / (-dx - dy) : 3 + dx / (dx - dy));
}
bool compareForSort1(const KriseevMTaskSeq::Point &origin, const KriseevMTaskSeq::Point &a,
                     const KriseevMTaskSeq::Point &b) {
  double angleA = angle1(origin, a);
  double angleB = angle1(origin, b);
  if (angleA < angleB) {
    return true;
  }
  if (angleA > angleB) {
    return false;
  }
  double dxA = a.first - origin.first;
  double dyA = a.second - origin.second;
  double dxB = b.first - origin.first;
  double dyB = b.second - origin.second;
  // The further point must be first
  // so the others will be ignored
  return dxA * dxA + dyA * dyA > dxB * dxB + dyB * dyB;
}

bool checkOrientation(const KriseevMTaskSeq::Point &origin, const KriseevMTaskSeq::Point &a,
                      const KriseevMTaskSeq::Point &b) {
  double dxA = a.first - origin.first;
  double dyA = a.second - origin.second;
  double dxB = b.first - origin.first;
  double dyB = b.second - origin.second;
  // Using cross product to check orientation
  return dxA * dyB - dyA * dxB < 0;
}

bool KriseevMTaskSeq::ConvexHullTask::pre_processing() {
  internal_order_test();
  auto *pointsX = reinterpret_cast<double *>(taskData->inputs.at(0));
  auto *pointsY = reinterpret_cast<double *>(taskData->inputs.at(1));
  points = std::vector<std::pair<double, double>>();
  points.reserve(taskData->inputs_count[0]);
  for (size_t i = 0; i < taskData->inputs_count[0]; ++i) {
    points.emplace_back(pointsX[i], pointsY[i]);
  }

  return true;
}

bool KriseevMTaskSeq::ConvexHullTask::validation() {
  internal_order_test();
  if (taskData->inputs_count.at(0) != taskData->inputs_count.at(1)) {
    return false;
  }
  if (taskData->inputs_count[0] < 3) {
    return false;
  }
  return true;
}

bool KriseevMTaskSeq::ConvexHullTask::run() {
  internal_order_test();
  auto originIt =
      std::min_element(points.begin(), points.end(), [](auto &a, auto &b) -> bool { return a.second < b.second; });
  auto origin = *originIt;

  for (size_t phase = 0; phase < points.size(); phase++) {
    if ((phase & 1) == 0) {
      for (size_t i = 1; i < points.size(); i += 2) {
        if (compareForSort1(origin, points[i], points[i - 1])) {
          std::iter_swap(points.begin() + i, points.begin() + i - 1);
        }
      }
    } else {
      for (size_t i = 1; i < points.size() - 1; i += 2) {
        if (compareForSort1(origin, points[i + 1], points[i])) {
          std::iter_swap(points.begin() + i, points.begin() + i + 1);
        }
      }
    }
  }

  std::vector<Point> hull;
  hull.reserve(points.size());
  hull.push_back(points[0]);
  hull.push_back(points[1]);
  hull.push_back(points[2]);

  for (uint32_t i = 3; i < points.size(); ++i) {
    while (hull.size() > 1 && !checkOrientation(hull.back(), *(hull.end() - 2), points[i])) {
      hull.pop_back();
    }
    hull.push_back(points[i]);
  }
  this->finalHull = hull;
  return true;
}

bool KriseevMTaskSeq::ConvexHullTask::post_processing() {
  internal_order_test();
  auto *resultsX = reinterpret_cast<double *>(taskData->outputs[0]);
  auto *resultsY = reinterpret_cast<double *>(taskData->outputs[1]);
  for (size_t i = 0; i < finalHull.size(); ++i) {
    resultsX[i] = finalHull[i].first;
    resultsY[i] = finalHull[i].second;
  }
  taskData->outputs_count[0] = finalHull.size();
  taskData->outputs_count[1] = finalHull.size();
  return true;
}
\end{lstlisting}

\section{Код версии с OpenMP}

\begin{lstlisting}
#include "include/kriseev_m_convex_hull_omp.hpp"

#include <algorithm>
#include <limits>

namespace KriseevMTaskOmp {

double angle(const KriseevMTaskOmp::Point &origin, const KriseevMTaskOmp::Point &point) {
  double dx = point.first - origin.first;
  double dy = point.second - origin.second;
  if (dx == 0.0 && dy == 0.0) {
    // Return this incorrect value to ensure
    // the origin is always first in sorted array
    return -1.0;
  }
  // Calculating angle using algorithm from
  // https://stackoverflow.com/a/14675998/14116050
  return (dy >= 0) ? (dx >= 0 ? dy / (dx + dy) : 1 - dx / (-dx + dy))
                   : (dx < 0 ? 2 - dy / (-dx - dy) : 3 + dx / (dx - dy));
}
bool compareForSort(const KriseevMTaskOmp::Point &origin, const KriseevMTaskOmp::Point &a,
                    const KriseevMTaskOmp::Point &b) {
  double angleA = angle(origin, a);
  double angleB = angle(origin, b);
  if (angleA < angleB) {
    return true;
  }
  if (angleA > angleB) {
    return false;
  }
  double dxA = a.first - origin.first;
  double dyA = a.second - origin.second;
  double dxB = b.first - origin.first;
  double dyB = b.second - origin.second;
  // The further point must be first
  // so the others will be ignored
  return dxA * dxA + dyA * dyA > dxB * dxB + dyB * dyB;
}

bool checkOrientation(const KriseevMTaskOmp::Point &origin, const KriseevMTaskOmp::Point &a,
                      const KriseevMTaskOmp::Point &b) {
  double dxA = a.first - origin.first;
  double dyA = a.second - origin.second;
  double dxB = b.first - origin.first;
  double dyB = b.second - origin.second;
  // Using cross product to check orientation
  return dxA * dyB - dyA * dxB < 0;
}

bool KriseevMTaskOmp::ConvexHullTask::pre_processing() {
  internal_order_test();
  auto *pointsX = reinterpret_cast<double *>(taskData->inputs.at(0));
  auto *pointsY = reinterpret_cast<double *>(taskData->inputs.at(1));
  points = std::vector<std::pair<double, double>>();
  points.reserve(taskData->inputs_count[0]);
  for (size_t i = 0; i < taskData->inputs_count[0]; ++i) {
    points.emplace_back(pointsX[i], pointsY[i]);
  }

  return true;
}

bool KriseevMTaskOmp::ConvexHullTask::validation() {
  internal_order_test();
  if (taskData->inputs_count.at(0) != taskData->inputs_count.at(1)) {
    return false;
  }
  if (taskData->inputs_count[0] < 3) {
    return false;
  }

  return true;
}

bool KriseevMTaskOmp::ConvexHullTask::run() {
  internal_order_test();
  auto originIt =
      std::min_element(points.begin(), points.end(), [](auto &a, auto &b) -> bool { return a.second < b.second; });
  auto origin = *originIt;

  int pointsSize = static_cast<int>(points.size());

#pragma omp parallel
  for (int phase = 0; phase < pointsSize; phase++) {
    if ((phase & 1) == 0) {
#pragma omp for
      for (int i = 1; i < pointsSize; i += 2) {
        if (compareForSort(origin, points[i], points[i - 1])) {
          std::iter_swap(points.begin() + i, points.begin() + i - 1);
        }
      }
    } else {
#pragma omp for
      for (int i = 1; i < pointsSize - 1; i += 2) {
        if (compareForSort(origin, points[i + 1], points[i])) {
          std::iter_swap(points.begin() + i, points.begin() + i + 1);
        }
      }
    }
  }

  std::vector<Point> hull;
  hull.reserve(points.size());
  hull.push_back(points[0]);
  hull.push_back(points[1]);
  hull.push_back(points[2]);

  for (uint32_t i = 3; i < points.size(); ++i) {
    while (hull.size() > 1 && !checkOrientation(hull.back(), *(hull.end() - 2), points[i])) {
      hull.pop_back();
    }
    hull.push_back(points[i]);
  }
  this->finalHull = hull;
  return true;
}

bool KriseevMTaskOmp::ConvexHullTask::post_processing() {
  internal_order_test();
  auto *resultsX = reinterpret_cast<double *>(taskData->outputs[0]);
  auto *resultsY = reinterpret_cast<double *>(taskData->outputs[1]);
  for (size_t i = 0; i < finalHull.size(); ++i) {
    resultsX[i] = finalHull[i].first;
    resultsY[i] = finalHull[i].second;
  }
  taskData->outputs_count[0] = finalHull.size();
  taskData->outputs_count[1] = finalHull.size();
  return true;
}

}  // namespace KriseevMTaskOmp
\end{lstlisting}

\section{Код версии с TBB}

\begin{lstlisting}
#include "include/kriseev_m_convex_hull_tbb.hpp"

#include <tbb/parallel_for.h>

#include <algorithm>
#include <limits>

namespace KriseevMTaskTbb {

double angle1(const KriseevMTaskTbb::Point &origin, const KriseevMTaskTbb::Point &point) {
  double dx = point.first - origin.first;
  double dy = point.second - origin.second;
  if (dx == 0.0 && dy == 0.0) {
    // Return this incorrect value to ensure 
    // the origin is always first in sorted array
    return -1.0;
  }
  // Calculating angle using algorithm from 
  // https://stackoverflow.com/a/14675998/14116050
  return (dy >= 0) ? (dx >= 0 ? dy / (dx + dy) : 1 - dx / (-dx + dy))
                   : (dx < 0 ? 2 - dy / (-dx - dy) : 3 + dx / (dx - dy));
}
bool compareForSort1(const KriseevMTaskTbb::Point &origin, const KriseevMTaskTbb::Point &a,
                     const KriseevMTaskTbb::Point &b) {
  double angleA = angle1(origin, a);
  double angleB = angle1(origin, b);
  if (angleA < angleB) {
    return true;
  }
  if (angleA > angleB) {
    return false;
  }
  double dxA = a.first - origin.first;
  double dyA = a.second - origin.second;
  double dxB = b.first - origin.first;
  double dyB = b.second - origin.second;
  // The further point must be first
  // so the others will be ignored
  return dxA * dxA + dyA * dyA > dxB * dxB + dyB * dyB;
}

bool checkOrientation(const KriseevMTaskTbb::Point &origin, const KriseevMTaskTbb::Point &a,
                      const KriseevMTaskTbb::Point &b) {
  double dxA = a.first - origin.first;
  double dyA = a.second - origin.second;
  double dxB = b.first - origin.first;
  double dyB = b.second - origin.second;
  // Using cross product to check orientation
  return dxA * dyB - dyA * dxB < 0;
}

bool KriseevMTaskTbb::ConvexHullTask::pre_processing() {
  internal_order_test();
  auto *pointsX = reinterpret_cast<double *>(taskData->inputs.at(0));
  auto *pointsY = reinterpret_cast<double *>(taskData->inputs.at(1));
  points = std::vector<std::pair<double, double>>();
  points.reserve(taskData->inputs_count[0]);
  for (size_t i = 0; i < taskData->inputs_count[0]; ++i) {
    points.emplace_back(pointsX[i], pointsY[i]);
  }

  return true;
}

bool KriseevMTaskTbb::ConvexHullTask::validation() {
  internal_order_test();
  if (taskData->inputs_count.at(0) != taskData->inputs_count.at(1)) {
    return false;
  }
  if (taskData->inputs_count[0] < 3) {
    return false;
  }
  return true;
}

bool KriseevMTaskTbb::ConvexHullTask::run() {
  internal_order_test();
  const size_t grainSizeDenominator = 16;
  auto originIt =
      std::min_element(points.begin(), points.end(), [](auto &a, auto &b) -> bool { return a.second < b.second; });
  auto origin = *originIt;
  for (size_t phase = 0; phase < points.size(); phase++) {
    if ((phase & 1) == 0) {
      tbb::parallel_for(tbb::blocked_range<size_t>(1, points.size(), points.size() / grainSizeDenominator),
                        [origin, this](const tbb::blocked_range<size_t> &r) {
                          size_t begin = r.begin();
                          const size_t end = r.end();
                          if ((begin & 1) == 0) {
                            begin++;
                          }
                          for (size_t i = begin; i < end; i += 2) {
                            if (compareForSort1(origin, points[i], points[i - 1])) {
                              std::iter_swap(points.begin() + i, points.begin() + i - 1);
                            }
                          }
                        });
    } else {
      tbb::parallel_for(tbb::blocked_range<size_t>(1, points.size() - 1, points.size() / grainSizeDenominator),
                        [origin, this](const tbb::blocked_range<size_t> &r) {
                          size_t begin = r.begin();
                          const size_t end = r.end();
                          if ((begin & 1) == 0) {
                            begin++;
                          }
                          for (size_t i = begin; i < end; i += 2) {
                            if (compareForSort1(origin, points[i + 1], points[i])) {
                              std::iter_swap(points.begin() + i, points.begin() + i + 1);
                            }
                          }
                        });
    }
  }

  std::vector<Point> hull;
  hull.reserve(points.size());
  hull.push_back(points[0]);
  hull.push_back(points[1]);
  hull.push_back(points[2]);

  for (uint32_t i = 3; i < points.size(); ++i) {
    while (hull.size() > 1 && !checkOrientation(hull.back(), *(hull.end() - 2), points[i])) {
      hull.pop_back();
    }
    hull.push_back(points[i]);
  }
  this->finalHull = hull;
  return true;
}

bool KriseevMTaskTbb::ConvexHullTask::post_processing() {
  internal_order_test();
  auto *resultsX = reinterpret_cast<double *>(taskData->outputs[0]);
  auto *resultsY = reinterpret_cast<double *>(taskData->outputs[1]);
  for (size_t i = 0; i < finalHull.size(); ++i) {
    resultsX[i] = finalHull[i].first;
    resultsY[i] = finalHull[i].second;
  }
  taskData->outputs_count[0] = finalHull.size();
  taskData->outputs_count[1] = finalHull.size();
  return true;
}

}  // namespace KriseevMTaskTbb
\end{lstlisting}

\section{Код версии с потоками STL}

\begin{lstlisting}
#include "include/kriseev_m_convex_hull_stl.hpp"

#include <algorithm>
#include <barrier>
#include <limits>
#include <thread>

namespace KriseevMTaskStl {

double angle1(const KriseevMTaskStl::Point &origin, const KriseevMTaskStl::Point &point) {
  double dx = point.first - origin.first;
  double dy = point.second - origin.second;
  if (dx == 0.0 && dy == 0.0) {
    // Return this incorrect value to ensure 
    // the origin is always first in sorted array
    return -1.0;
  }
  // Calculating angle using algorithm 
  // from https://stackoverflow.com/a/14675998/14116050
  return (dx >= 0 ? dy / (dx + dy) : 1 - dx / (dy - dx));
}
bool compareForSort1(const KriseevMTaskStl::Point &origin, const KriseevMTaskStl::Point &a,
                     const KriseevMTaskStl::Point &b) {
  double angleA = angle1(origin, a);
  double angleB = angle1(origin, b);
  if (angleA < angleB) {
    return true;
  }
  if (angleA > angleB) {
    return false;
  }
  double dxA = a.first - origin.first;
  double dyA = a.second - origin.second;
  double dxB = b.first - origin.first;
  double dyB = b.second - origin.second;
  // The further point must be first
  // so the others will be ignored
  return dxA * dxA + dyA * dyA > dxB * dxB + dyB * dyB;
}

void sortPoints(std::vector<Point> &points, const KriseevMTaskStl::Point &origin) {
  size_t numThreads = std::thread::hardware_concurrency();

  size_t chunkSize = points.size() / numThreads;
  if (points.size() % numThreads > 0) {
    chunkSize++;
  }
  if (points.size() <= numThreads) {
    chunkSize = points.size();
    numThreads = 1;
  }

  std::vector<std::thread> threads;
  std::barrier b(numThreads);

  auto work = [&b](std::vector<Point> &points, Point origin, size_t chunkSize, size_t threadNum) {
    size_t firstIndex = threadNum * chunkSize;
    if ((firstIndex & 1) == 0) firstIndex++;

    size_t lastIndex = (threadNum + 1) * chunkSize;
    if (lastIndex > points.size()) {
      lastIndex = points.size();
    }
    size_t pointsSize = points.size();
    auto pointsBegin = points.begin();
    
    size_t lastIndexOdd = lastIndex == points.size() ? lastIndex - 1 : lastIndex;
    for (size_t phase = 0; phase < pointsSize; phase++) {
      if ((phase & 1) == 0) {
        for (size_t i = firstIndex; i < lastIndex; i += 2) {
          if (compareForSort1(origin, points[i], points[i - 1])) {
            std::iter_swap(pointsBegin + i, pointsBegin + i - 1);
          }
        }
      } else {
        for (size_t i = firstIndex; i < lastIndexOdd; i += 2) {
          if (compareForSort1(origin, points[i + 1], points[i])) {
            std::iter_swap(pointsBegin + i, pointsBegin + i + 1);
          }
        }
      }

      b.arrive_and_wait();
    }
  };
  for (size_t t = 0; t < numThreads; ++t) {
    threads.emplace_back(work, std::ref(points), origin, chunkSize, t);
  }
  for (size_t t = 0; t < numThreads; ++t) {
    threads[t].join();
  }
}

bool checkOrientation(const KriseevMTaskStl::Point &origin, const KriseevMTaskStl::Point &a,
                      const KriseevMTaskStl::Point &b) {
  double dxA = a.first - origin.first;
  double dyA = a.second - origin.second;
  double dxB = b.first - origin.first;
  double dyB = b.second - origin.second;
  // Using cross product to check orientation
  return dxA * dyB - dyA * dxB < 0;
}

bool KriseevMTaskStl::ConvexHullTask::pre_processing() {
  internal_order_test();
  auto *pointsX = reinterpret_cast<double *>(taskData->inputs.at(0));
  auto *pointsY = reinterpret_cast<double *>(taskData->inputs.at(1));
  points = std::vector<std::pair<double, double>>();
  points.reserve(taskData->inputs_count[0]);
  for (size_t i = 0; i < taskData->inputs_count[0]; ++i) {
    points.emplace_back(pointsX[i], pointsY[i]);
  }

  return true;
}

bool KriseevMTaskStl::ConvexHullTask::validation() {
  internal_order_test();
  if (taskData->inputs_count.at(0) != taskData->inputs_count.at(1)) {
    return false;
  }
  if (taskData->inputs_count[0] < 3) {
    return false;
  }
  return true;
}

bool KriseevMTaskStl::ConvexHullTask::run() {
  internal_order_test();
  auto originIt =
      std::min_element(points.begin(), points.end(), [](auto &a, auto &b) -> bool { return a.second < b.second; });
  auto origin = *originIt;

  sortPoints(points, origin);

  std::vector<Point> hull;
  hull.reserve(points.size());
  hull.push_back(points[0]);
  hull.push_back(points[1]);
  hull.push_back(points[2]);

  for (uint32_t i = 3; i < points.size(); ++i) {
    while (hull.size() > 1 && !checkOrientation(hull.back(), *(hull.end() - 2), points[i])) {
      hull.pop_back();
    }
    hull.push_back(points[i]);
  }
  this->finalHull = hull;
  return true;
}
bool KriseevMTaskStl::ConvexHullTask::post_processing() {
  internal_order_test();
  auto *resultsX = reinterpret_cast<double *>(taskData->outputs[0]);
  auto *resultsY = reinterpret_cast<double *>(taskData->outputs[1]);
  for (size_t i = 0; i < finalHull.size(); ++i) {
    resultsX[i] = finalHull[i].first;
    resultsY[i] = finalHull[i].second;
  }
  taskData->outputs_count[0] = finalHull.size();
  taskData->outputs_count[1] = finalHull.size();
  return true;
}

}  // namespace KriseevMTaskStl
\end{lstlisting}

\section{Код Performance тестов}

Тесты для фреймворка Google Tests. Код для каждой из версий отличается 
только используемым пространством имён и таймером для соответствующей технологии.

\begin{lstlisting}
TEST(kriseev_m_convex_hull_graham_stl, test_pipeline_run) {
  const int count = 5000;
  // Create data
  std::vector<double> pX;
  std::vector<double> pY;
  std::mt19937 gen(1587443);
  std::uniform_real_distribution<> dis(-2.5, 4.0);
  for (int i = 0; i < count; ++i) {
    pX.emplace_back(dis(gen));
    pY.emplace_back(dis(gen));
  }
  std::vector<std::pair<double, double>> expectedPoints = {
      {-3.0, -3.5},
      {4.8, -3.2},
      {4.65, 4.5},
      {-3.2, 4.25},
  };
  pX.emplace_back(expectedPoints[1].first);
  pY.emplace_back(expectedPoints[1].second);
  pX.emplace_back(expectedPoints[0].first);
  pY.emplace_back(expectedPoints[0].second);
  pX.emplace_back(expectedPoints[3].first);
  pY.emplace_back(expectedPoints[3].second);
  pX.emplace_back(expectedPoints[2].first);
  pY.emplace_back(expectedPoints[2].second);
  std::vector<double> outX(pX.size());
  std::vector<double> outY(pY.size());

  // Create TaskData
  std::shared_ptr<ppc::core::TaskData> data = std::make_shared<ppc::core::TaskData>();
  data->inputs.emplace_back(reinterpret_cast<uint8_t *>(pX.data()));
  data->inputs_count.emplace_back(pX.size());
  data->inputs.emplace_back(reinterpret_cast<uint8_t *>(pY.data()));
  data->inputs_count.emplace_back(pY.size());

  data->outputs.emplace_back(reinterpret_cast<uint8_t *>(outX.data()));
  data->outputs_count.emplace_back(outX.size());
  data->outputs.emplace_back(reinterpret_cast<uint8_t *>(outY.data()));
  data->outputs_count.emplace_back(outY.size());

  // Create Task
  auto sequentialTask = std::make_shared<KriseevMTaskStl::ConvexHullTask>(data);

  // Create Perf attributes
  auto perfAttr = std::make_shared<ppc::core::PerfAttr>();
  perfAttr->num_running = 10;

  const auto t0 = std::chrono::high_resolution_clock::now();
  perfAttr->current_timer = [&] {
    auto current_time_point = std::chrono::high_resolution_clock::now();
    auto duration = std::chrono::duration_cast<std::chrono::nanoseconds>(
        current_time_point - t0).count();
    return static_cast<double>(duration) * 1e-9;
  };

  // Create and init perf results
  auto perfResults = std::make_shared<ppc::core::PerfResults>();

  // Create Perf analyzer
  auto perfAnalyzer = std::make_shared<ppc::core::Perf>(sequentialTask);
  perfAnalyzer->pipeline_run(perfAttr, perfResults);
  ppc::core::Perf::print_perf_statistic(perfResults);

  ASSERT_EQ(expectedPoints.size(), data->outputs_count[0]);
  ASSERT_EQ(expectedPoints.size(), data->outputs_count[1]);

  for (size_t i = 0; i < expectedPoints.size(); ++i) {
    EXPECT_EQ(expectedPoints[i].first, outX[i]);
    EXPECT_EQ(expectedPoints[i].second, outY[i]);
  }
}
TEST(kriseev_m_convex_hull_graham_stl, test_task_run) {
  const int count = 5000;
  // Create data
  std::vector<double> pX;
  std::vector<double> pY;
  std::mt19937 gen(1589938);
  std::uniform_real_distribution<> dis(-2.5, 4.0);
  for (int i = 0; i < count; ++i) {
    pX.emplace_back(dis(gen));
    pY.emplace_back(dis(gen));
  }
  std::vector<std::pair<double, double>> expectedPoints = {
      {-3.0, -3.5},
      {4.8, -3.2},
      {4.65, 4.5},
      {-3.2, 4.25},
  };
  pX.emplace_back(expectedPoints[1].first);
  pY.emplace_back(expectedPoints[1].second);
  pX.emplace_back(expectedPoints[0].first);
  pY.emplace_back(expectedPoints[0].second);
  pX.emplace_back(expectedPoints[3].first);
  pY.emplace_back(expectedPoints[3].second);
  pX.emplace_back(expectedPoints[2].first);
  pY.emplace_back(expectedPoints[2].second);
  std::vector<double> outX(pX.size());
  std::vector<double> outY(pY.size());

  // Create TaskData
  std::shared_ptr<ppc::core::TaskData> data = std::make_shared<ppc::core::TaskData>();
  data->inputs.emplace_back(reinterpret_cast<uint8_t *>(pX.data()));
  data->inputs_count.emplace_back(pX.size());
  data->inputs.emplace_back(reinterpret_cast<uint8_t *>(pY.data()));
  data->inputs_count.emplace_back(pY.size());

  data->outputs.emplace_back(reinterpret_cast<uint8_t *>(outX.data()));
  data->outputs_count.emplace_back(outX.size());
  data->outputs.emplace_back(reinterpret_cast<uint8_t *>(outY.data()));
  data->outputs_count.emplace_back(outY.size());

  // Create Task
  auto sequentialTask = std::make_shared<KriseevMTaskStl::ConvexHullTask>(data);

  // Create Perf attributes
  auto perfAttr = std::make_shared<ppc::core::PerfAttr>();
  perfAttr->num_running = 10;
  const auto t0 = std::chrono::high_resolution_clock::now();
  perfAttr->current_timer = [&] {
    auto current_time_point = std::chrono::high_resolution_clock::now();
    auto duration = std::chrono::duration_cast<std::chrono::nanoseconds>(
        current_time_point - t0).count();
    return static_cast<double>(duration) * 1e-9;
  };

  // Create and init perf results
  auto perfResults = std::make_shared<ppc::core::PerfResults>();

  // Create Perf analyzer
  auto perfAnalyzer = std::make_shared<ppc::core::Perf>(sequentialTask);
  perfAnalyzer->task_run(perfAttr, perfResults);
  ppc::core::Perf::print_perf_statistic(perfResults);

  ASSERT_EQ(expectedPoints.size(), data->outputs_count[0]);
  ASSERT_EQ(expectedPoints.size(), data->outputs_count[1]);

  for (size_t i = 0; i < expectedPoints.size(); ++i) {
    EXPECT_EQ(expectedPoints[i].first, outX[i]);
    EXPECT_EQ(expectedPoints[i].second, outY[i]);
  }
}

\end{lstlisting}

\end{document}
